
% Default to the notebook output style

    


% Inherit from the specified cell style.




    
\documentclass[11pt]{article}

    
    
    \usepackage[T1]{fontenc}
    % Nicer default font (+ math font) than Computer Modern for most use cases
    \usepackage{mathpazo}

    % Basic figure setup, for now with no caption control since it's done
    % automatically by Pandoc (which extracts ![](path) syntax from Markdown).
    \usepackage{graphicx}
    % We will generate all images so they have a width \maxwidth. This means
    % that they will get their normal width if they fit onto the page, but
    % are scaled down if they would overflow the margins.
    \makeatletter
    \def\maxwidth{\ifdim\Gin@nat@width>\linewidth\linewidth
    \else\Gin@nat@width\fi}
    \makeatother
    \let\Oldincludegraphics\includegraphics
    % Set max figure width to be 80% of text width, for now hardcoded.
    \renewcommand{\includegraphics}[1]{\Oldincludegraphics[width=.8\maxwidth]{#1}}
    % Ensure that by default, figures have no caption (until we provide a
    % proper Figure object with a Caption API and a way to capture that
    % in the conversion process - todo).
    \usepackage{caption}
    \DeclareCaptionLabelFormat{nolabel}{}
    \captionsetup{labelformat=nolabel}

    \usepackage{adjustbox} % Used to constrain images to a maximum size 
    \usepackage{xcolor} % Allow colors to be defined
    \usepackage{enumerate} % Needed for markdown enumerations to work
    \usepackage{geometry} % Used to adjust the document margins
    \usepackage{amsmath} % Equations
    \usepackage{amssymb} % Equations
    \usepackage{textcomp} % defines textquotesingle
    % Hack from http://tex.stackexchange.com/a/47451/13684:
    \AtBeginDocument{%
        \def\PYZsq{\textquotesingle}% Upright quotes in Pygmentized code
    }
    \usepackage{upquote} % Upright quotes for verbatim code
    \usepackage{eurosym} % defines \euro
    \usepackage[mathletters]{ucs} % Extended unicode (utf-8) support
    \usepackage[utf8x]{inputenc} % Allow utf-8 characters in the tex document
    \usepackage{fancyvrb} % verbatim replacement that allows latex
    \usepackage{grffile} % extends the file name processing of package graphics 
                         % to support a larger range 
    % The hyperref package gives us a pdf with properly built
    % internal navigation ('pdf bookmarks' for the table of contents,
    % internal cross-reference links, web links for URLs, etc.)
    \usepackage{hyperref}
    \usepackage{longtable} % longtable support required by pandoc >1.10
    \usepackage{booktabs}  % table support for pandoc > 1.12.2
    \usepackage[inline]{enumitem} % IRkernel/repr support (it uses the enumerate* environment)
    \usepackage[normalem]{ulem} % ulem is needed to support strikethroughs (\sout)
                                % normalem makes italics be italics, not underlines
    

    
    
    % Colors for the hyperref package
    \definecolor{urlcolor}{rgb}{0,.145,.698}
    \definecolor{linkcolor}{rgb}{.71,0.21,0.01}
    \definecolor{citecolor}{rgb}{.12,.54,.11}

    % ANSI colors
    \definecolor{ansi-black}{HTML}{3E424D}
    \definecolor{ansi-black-intense}{HTML}{282C36}
    \definecolor{ansi-red}{HTML}{E75C58}
    \definecolor{ansi-red-intense}{HTML}{B22B31}
    \definecolor{ansi-green}{HTML}{00A250}
    \definecolor{ansi-green-intense}{HTML}{007427}
    \definecolor{ansi-yellow}{HTML}{DDB62B}
    \definecolor{ansi-yellow-intense}{HTML}{B27D12}
    \definecolor{ansi-blue}{HTML}{208FFB}
    \definecolor{ansi-blue-intense}{HTML}{0065CA}
    \definecolor{ansi-magenta}{HTML}{D160C4}
    \definecolor{ansi-magenta-intense}{HTML}{A03196}
    \definecolor{ansi-cyan}{HTML}{60C6C8}
    \definecolor{ansi-cyan-intense}{HTML}{258F8F}
    \definecolor{ansi-white}{HTML}{C5C1B4}
    \definecolor{ansi-white-intense}{HTML}{A1A6B2}

    % commands and environments needed by pandoc snippets
    % extracted from the output of `pandoc -s`
    \providecommand{\tightlist}{%
      \setlength{\itemsep}{0pt}\setlength{\parskip}{0pt}}
    \DefineVerbatimEnvironment{Highlighting}{Verbatim}{commandchars=\\\{\}}
    % Add ',fontsize=\small' for more characters per line
    \newenvironment{Shaded}{}{}
    \newcommand{\KeywordTok}[1]{\textcolor[rgb]{0.00,0.44,0.13}{\textbf{{#1}}}}
    \newcommand{\DataTypeTok}[1]{\textcolor[rgb]{0.56,0.13,0.00}{{#1}}}
    \newcommand{\DecValTok}[1]{\textcolor[rgb]{0.25,0.63,0.44}{{#1}}}
    \newcommand{\BaseNTok}[1]{\textcolor[rgb]{0.25,0.63,0.44}{{#1}}}
    \newcommand{\FloatTok}[1]{\textcolor[rgb]{0.25,0.63,0.44}{{#1}}}
    \newcommand{\CharTok}[1]{\textcolor[rgb]{0.25,0.44,0.63}{{#1}}}
    \newcommand{\StringTok}[1]{\textcolor[rgb]{0.25,0.44,0.63}{{#1}}}
    \newcommand{\CommentTok}[1]{\textcolor[rgb]{0.38,0.63,0.69}{\textit{{#1}}}}
    \newcommand{\OtherTok}[1]{\textcolor[rgb]{0.00,0.44,0.13}{{#1}}}
    \newcommand{\AlertTok}[1]{\textcolor[rgb]{1.00,0.00,0.00}{\textbf{{#1}}}}
    \newcommand{\FunctionTok}[1]{\textcolor[rgb]{0.02,0.16,0.49}{{#1}}}
    \newcommand{\RegionMarkerTok}[1]{{#1}}
    \newcommand{\ErrorTok}[1]{\textcolor[rgb]{1.00,0.00,0.00}{\textbf{{#1}}}}
    \newcommand{\NormalTok}[1]{{#1}}
    
    % Additional commands for more recent versions of Pandoc
    \newcommand{\ConstantTok}[1]{\textcolor[rgb]{0.53,0.00,0.00}{{#1}}}
    \newcommand{\SpecialCharTok}[1]{\textcolor[rgb]{0.25,0.44,0.63}{{#1}}}
    \newcommand{\VerbatimStringTok}[1]{\textcolor[rgb]{0.25,0.44,0.63}{{#1}}}
    \newcommand{\SpecialStringTok}[1]{\textcolor[rgb]{0.73,0.40,0.53}{{#1}}}
    \newcommand{\ImportTok}[1]{{#1}}
    \newcommand{\DocumentationTok}[1]{\textcolor[rgb]{0.73,0.13,0.13}{\textit{{#1}}}}
    \newcommand{\AnnotationTok}[1]{\textcolor[rgb]{0.38,0.63,0.69}{\textbf{\textit{{#1}}}}}
    \newcommand{\CommentVarTok}[1]{\textcolor[rgb]{0.38,0.63,0.69}{\textbf{\textit{{#1}}}}}
    \newcommand{\VariableTok}[1]{\textcolor[rgb]{0.10,0.09,0.49}{{#1}}}
    \newcommand{\ControlFlowTok}[1]{\textcolor[rgb]{0.00,0.44,0.13}{\textbf{{#1}}}}
    \newcommand{\OperatorTok}[1]{\textcolor[rgb]{0.40,0.40,0.40}{{#1}}}
    \newcommand{\BuiltInTok}[1]{{#1}}
    \newcommand{\ExtensionTok}[1]{{#1}}
    \newcommand{\PreprocessorTok}[1]{\textcolor[rgb]{0.74,0.48,0.00}{{#1}}}
    \newcommand{\AttributeTok}[1]{\textcolor[rgb]{0.49,0.56,0.16}{{#1}}}
    \newcommand{\InformationTok}[1]{\textcolor[rgb]{0.38,0.63,0.69}{\textbf{\textit{{#1}}}}}
    \newcommand{\WarningTok}[1]{\textcolor[rgb]{0.38,0.63,0.69}{\textbf{\textit{{#1}}}}}
    
    
    % Define a nice break command that doesn't care if a line doesn't already
    % exist.
    \def\br{\hspace*{\fill} \\* }
    % Math Jax compatability definitions
    \def\gt{>}
    \def\lt{<}
    % Document parameters
    \title{Maharjan\_Homework2\_List\_and\_Tuples\_Tutorial\_Fall2018}
    
    
    

    % Pygments definitions
    
\makeatletter
\def\PY@reset{\let\PY@it=\relax \let\PY@bf=\relax%
    \let\PY@ul=\relax \let\PY@tc=\relax%
    \let\PY@bc=\relax \let\PY@ff=\relax}
\def\PY@tok#1{\csname PY@tok@#1\endcsname}
\def\PY@toks#1+{\ifx\relax#1\empty\else%
    \PY@tok{#1}\expandafter\PY@toks\fi}
\def\PY@do#1{\PY@bc{\PY@tc{\PY@ul{%
    \PY@it{\PY@bf{\PY@ff{#1}}}}}}}
\def\PY#1#2{\PY@reset\PY@toks#1+\relax+\PY@do{#2}}

\expandafter\def\csname PY@tok@gd\endcsname{\def\PY@tc##1{\textcolor[rgb]{0.63,0.00,0.00}{##1}}}
\expandafter\def\csname PY@tok@gu\endcsname{\let\PY@bf=\textbf\def\PY@tc##1{\textcolor[rgb]{0.50,0.00,0.50}{##1}}}
\expandafter\def\csname PY@tok@gt\endcsname{\def\PY@tc##1{\textcolor[rgb]{0.00,0.27,0.87}{##1}}}
\expandafter\def\csname PY@tok@gs\endcsname{\let\PY@bf=\textbf}
\expandafter\def\csname PY@tok@gr\endcsname{\def\PY@tc##1{\textcolor[rgb]{1.00,0.00,0.00}{##1}}}
\expandafter\def\csname PY@tok@cm\endcsname{\let\PY@it=\textit\def\PY@tc##1{\textcolor[rgb]{0.25,0.50,0.50}{##1}}}
\expandafter\def\csname PY@tok@vg\endcsname{\def\PY@tc##1{\textcolor[rgb]{0.10,0.09,0.49}{##1}}}
\expandafter\def\csname PY@tok@vi\endcsname{\def\PY@tc##1{\textcolor[rgb]{0.10,0.09,0.49}{##1}}}
\expandafter\def\csname PY@tok@vm\endcsname{\def\PY@tc##1{\textcolor[rgb]{0.10,0.09,0.49}{##1}}}
\expandafter\def\csname PY@tok@mh\endcsname{\def\PY@tc##1{\textcolor[rgb]{0.40,0.40,0.40}{##1}}}
\expandafter\def\csname PY@tok@cs\endcsname{\let\PY@it=\textit\def\PY@tc##1{\textcolor[rgb]{0.25,0.50,0.50}{##1}}}
\expandafter\def\csname PY@tok@ge\endcsname{\let\PY@it=\textit}
\expandafter\def\csname PY@tok@vc\endcsname{\def\PY@tc##1{\textcolor[rgb]{0.10,0.09,0.49}{##1}}}
\expandafter\def\csname PY@tok@il\endcsname{\def\PY@tc##1{\textcolor[rgb]{0.40,0.40,0.40}{##1}}}
\expandafter\def\csname PY@tok@go\endcsname{\def\PY@tc##1{\textcolor[rgb]{0.53,0.53,0.53}{##1}}}
\expandafter\def\csname PY@tok@cp\endcsname{\def\PY@tc##1{\textcolor[rgb]{0.74,0.48,0.00}{##1}}}
\expandafter\def\csname PY@tok@gi\endcsname{\def\PY@tc##1{\textcolor[rgb]{0.00,0.63,0.00}{##1}}}
\expandafter\def\csname PY@tok@gh\endcsname{\let\PY@bf=\textbf\def\PY@tc##1{\textcolor[rgb]{0.00,0.00,0.50}{##1}}}
\expandafter\def\csname PY@tok@ni\endcsname{\let\PY@bf=\textbf\def\PY@tc##1{\textcolor[rgb]{0.60,0.60,0.60}{##1}}}
\expandafter\def\csname PY@tok@nl\endcsname{\def\PY@tc##1{\textcolor[rgb]{0.63,0.63,0.00}{##1}}}
\expandafter\def\csname PY@tok@nn\endcsname{\let\PY@bf=\textbf\def\PY@tc##1{\textcolor[rgb]{0.00,0.00,1.00}{##1}}}
\expandafter\def\csname PY@tok@no\endcsname{\def\PY@tc##1{\textcolor[rgb]{0.53,0.00,0.00}{##1}}}
\expandafter\def\csname PY@tok@na\endcsname{\def\PY@tc##1{\textcolor[rgb]{0.49,0.56,0.16}{##1}}}
\expandafter\def\csname PY@tok@nb\endcsname{\def\PY@tc##1{\textcolor[rgb]{0.00,0.50,0.00}{##1}}}
\expandafter\def\csname PY@tok@nc\endcsname{\let\PY@bf=\textbf\def\PY@tc##1{\textcolor[rgb]{0.00,0.00,1.00}{##1}}}
\expandafter\def\csname PY@tok@nd\endcsname{\def\PY@tc##1{\textcolor[rgb]{0.67,0.13,1.00}{##1}}}
\expandafter\def\csname PY@tok@ne\endcsname{\let\PY@bf=\textbf\def\PY@tc##1{\textcolor[rgb]{0.82,0.25,0.23}{##1}}}
\expandafter\def\csname PY@tok@nf\endcsname{\def\PY@tc##1{\textcolor[rgb]{0.00,0.00,1.00}{##1}}}
\expandafter\def\csname PY@tok@si\endcsname{\let\PY@bf=\textbf\def\PY@tc##1{\textcolor[rgb]{0.73,0.40,0.53}{##1}}}
\expandafter\def\csname PY@tok@s2\endcsname{\def\PY@tc##1{\textcolor[rgb]{0.73,0.13,0.13}{##1}}}
\expandafter\def\csname PY@tok@nt\endcsname{\let\PY@bf=\textbf\def\PY@tc##1{\textcolor[rgb]{0.00,0.50,0.00}{##1}}}
\expandafter\def\csname PY@tok@nv\endcsname{\def\PY@tc##1{\textcolor[rgb]{0.10,0.09,0.49}{##1}}}
\expandafter\def\csname PY@tok@s1\endcsname{\def\PY@tc##1{\textcolor[rgb]{0.73,0.13,0.13}{##1}}}
\expandafter\def\csname PY@tok@dl\endcsname{\def\PY@tc##1{\textcolor[rgb]{0.73,0.13,0.13}{##1}}}
\expandafter\def\csname PY@tok@ch\endcsname{\let\PY@it=\textit\def\PY@tc##1{\textcolor[rgb]{0.25,0.50,0.50}{##1}}}
\expandafter\def\csname PY@tok@m\endcsname{\def\PY@tc##1{\textcolor[rgb]{0.40,0.40,0.40}{##1}}}
\expandafter\def\csname PY@tok@gp\endcsname{\let\PY@bf=\textbf\def\PY@tc##1{\textcolor[rgb]{0.00,0.00,0.50}{##1}}}
\expandafter\def\csname PY@tok@sh\endcsname{\def\PY@tc##1{\textcolor[rgb]{0.73,0.13,0.13}{##1}}}
\expandafter\def\csname PY@tok@ow\endcsname{\let\PY@bf=\textbf\def\PY@tc##1{\textcolor[rgb]{0.67,0.13,1.00}{##1}}}
\expandafter\def\csname PY@tok@sx\endcsname{\def\PY@tc##1{\textcolor[rgb]{0.00,0.50,0.00}{##1}}}
\expandafter\def\csname PY@tok@bp\endcsname{\def\PY@tc##1{\textcolor[rgb]{0.00,0.50,0.00}{##1}}}
\expandafter\def\csname PY@tok@c1\endcsname{\let\PY@it=\textit\def\PY@tc##1{\textcolor[rgb]{0.25,0.50,0.50}{##1}}}
\expandafter\def\csname PY@tok@fm\endcsname{\def\PY@tc##1{\textcolor[rgb]{0.00,0.00,1.00}{##1}}}
\expandafter\def\csname PY@tok@o\endcsname{\def\PY@tc##1{\textcolor[rgb]{0.40,0.40,0.40}{##1}}}
\expandafter\def\csname PY@tok@kc\endcsname{\let\PY@bf=\textbf\def\PY@tc##1{\textcolor[rgb]{0.00,0.50,0.00}{##1}}}
\expandafter\def\csname PY@tok@c\endcsname{\let\PY@it=\textit\def\PY@tc##1{\textcolor[rgb]{0.25,0.50,0.50}{##1}}}
\expandafter\def\csname PY@tok@mf\endcsname{\def\PY@tc##1{\textcolor[rgb]{0.40,0.40,0.40}{##1}}}
\expandafter\def\csname PY@tok@err\endcsname{\def\PY@bc##1{\setlength{\fboxsep}{0pt}\fcolorbox[rgb]{1.00,0.00,0.00}{1,1,1}{\strut ##1}}}
\expandafter\def\csname PY@tok@mb\endcsname{\def\PY@tc##1{\textcolor[rgb]{0.40,0.40,0.40}{##1}}}
\expandafter\def\csname PY@tok@ss\endcsname{\def\PY@tc##1{\textcolor[rgb]{0.10,0.09,0.49}{##1}}}
\expandafter\def\csname PY@tok@sr\endcsname{\def\PY@tc##1{\textcolor[rgb]{0.73,0.40,0.53}{##1}}}
\expandafter\def\csname PY@tok@mo\endcsname{\def\PY@tc##1{\textcolor[rgb]{0.40,0.40,0.40}{##1}}}
\expandafter\def\csname PY@tok@kd\endcsname{\let\PY@bf=\textbf\def\PY@tc##1{\textcolor[rgb]{0.00,0.50,0.00}{##1}}}
\expandafter\def\csname PY@tok@mi\endcsname{\def\PY@tc##1{\textcolor[rgb]{0.40,0.40,0.40}{##1}}}
\expandafter\def\csname PY@tok@kn\endcsname{\let\PY@bf=\textbf\def\PY@tc##1{\textcolor[rgb]{0.00,0.50,0.00}{##1}}}
\expandafter\def\csname PY@tok@cpf\endcsname{\let\PY@it=\textit\def\PY@tc##1{\textcolor[rgb]{0.25,0.50,0.50}{##1}}}
\expandafter\def\csname PY@tok@kr\endcsname{\let\PY@bf=\textbf\def\PY@tc##1{\textcolor[rgb]{0.00,0.50,0.00}{##1}}}
\expandafter\def\csname PY@tok@s\endcsname{\def\PY@tc##1{\textcolor[rgb]{0.73,0.13,0.13}{##1}}}
\expandafter\def\csname PY@tok@kp\endcsname{\def\PY@tc##1{\textcolor[rgb]{0.00,0.50,0.00}{##1}}}
\expandafter\def\csname PY@tok@w\endcsname{\def\PY@tc##1{\textcolor[rgb]{0.73,0.73,0.73}{##1}}}
\expandafter\def\csname PY@tok@kt\endcsname{\def\PY@tc##1{\textcolor[rgb]{0.69,0.00,0.25}{##1}}}
\expandafter\def\csname PY@tok@sc\endcsname{\def\PY@tc##1{\textcolor[rgb]{0.73,0.13,0.13}{##1}}}
\expandafter\def\csname PY@tok@sb\endcsname{\def\PY@tc##1{\textcolor[rgb]{0.73,0.13,0.13}{##1}}}
\expandafter\def\csname PY@tok@sa\endcsname{\def\PY@tc##1{\textcolor[rgb]{0.73,0.13,0.13}{##1}}}
\expandafter\def\csname PY@tok@k\endcsname{\let\PY@bf=\textbf\def\PY@tc##1{\textcolor[rgb]{0.00,0.50,0.00}{##1}}}
\expandafter\def\csname PY@tok@se\endcsname{\let\PY@bf=\textbf\def\PY@tc##1{\textcolor[rgb]{0.73,0.40,0.13}{##1}}}
\expandafter\def\csname PY@tok@sd\endcsname{\let\PY@it=\textit\def\PY@tc##1{\textcolor[rgb]{0.73,0.13,0.13}{##1}}}

\def\PYZbs{\char`\\}
\def\PYZus{\char`\_}
\def\PYZob{\char`\{}
\def\PYZcb{\char`\}}
\def\PYZca{\char`\^}
\def\PYZam{\char`\&}
\def\PYZlt{\char`\<}
\def\PYZgt{\char`\>}
\def\PYZsh{\char`\#}
\def\PYZpc{\char`\%}
\def\PYZdl{\char`\$}
\def\PYZhy{\char`\-}
\def\PYZsq{\char`\'}
\def\PYZdq{\char`\"}
\def\PYZti{\char`\~}
% for compatibility with earlier versions
\def\PYZat{@}
\def\PYZlb{[}
\def\PYZrb{]}
\makeatother


    % Exact colors from NB
    \definecolor{incolor}{rgb}{0.0, 0.0, 0.5}
    \definecolor{outcolor}{rgb}{0.545, 0.0, 0.0}



    
    % Prevent overflowing lines due to hard-to-break entities
    \sloppy 
    % Setup hyperref package
    \hypersetup{
      breaklinks=true,  % so long urls are correctly broken across lines
      colorlinks=true,
      urlcolor=urlcolor,
      linkcolor=linkcolor,
      citecolor=citecolor,
      }
    % Slightly bigger margins than the latex defaults
    
    \geometry{verbose,tmargin=1in,bmargin=1in,lmargin=1in,rmargin=1in}
    
    

    \begin{document}
    
    
    \maketitle
    
    

    
    \section{Tutorial: Difference between Lists and
Tuples}\label{tutorial-difference-between-lists-and-tuples}

The following content and code excerpts have been derived from the
following sources. 1. Python Tutorial from Tutorialspoint.
https://www.tutorialspoint.com/python/ 2. How to Loop with indexes in
Python.
http://treyhunner.com/2016/04/how-to-loop-with-indexes-in-python/ 3.
Python.org Tutorials. https://docs.python.org/2/tutorial/index.html 4.
Learnpython.org. https://www.learnpython.org 5. Introductin to Python
Programming
https://github.com/jrjohansson/scientific-python-lectures/blob/master/Lecture-1-Introduction-to-Python-Programming.ipynb
6. Iterating with Python Lambdas.
https://caisbalderas.com/blog/iterating-with-python-lambdas/ 7. Proper
way to copy lists. http://henry.precheur.org/python/copy\_list.html

    \subsection{Creation and Assignment of Lists and
Tuples}\label{creation-and-assignment-of-lists-and-tuples}

    "Lists are what they seem - a list of values. Each one of them is
numbered, starting from zero - the first one is numbered zero, the
second 1, the third 2, etc. You can remove values from the list, and add
new values to the end."

Lists are assigned using square brackets and values are separate with
commas

Syntax for Lists in Python is {[} . . . {]}

    \begin{Verbatim}[commandchars=\\\{\}]
{\color{incolor}In [{\color{incolor}19}]:} \PY{n}{b} \PY{o}{=} \PY{p}{[}\PY{l+m+mi}{1}\PY{p}{,} \PY{l+m+mi}{2}\PY{p}{,} \PY{l+m+mi}{3}\PY{p}{]} \PY{c+c1}{\PYZsh{}lists use square brackets for assignment}
         
         \PY{k}{print}\PY{p}{(}\PY{n+nb}{type}\PY{p}{(}\PY{n}{b}\PY{p}{)}\PY{p}{)}
         \PY{k}{print}\PY{p}{(}\PY{n}{b}\PY{p}{)}
\end{Verbatim}


    \begin{Verbatim}[commandchars=\\\{\}]
<type 'list'>
[1, 2, 3]

    \end{Verbatim}

    "Tuples are just like lists, but you can't change their values. The
values that you give it first up, are the values that you are stuck with
for the rest of the program. Again, each value is numbered starting from
zero, for easy reference."

Tuples are assigned using rounded parantheses and values are separated
with commas

Syntax for Lists in Python is ( . . . )

    \begin{Verbatim}[commandchars=\\\{\}]
{\color{incolor}In [{\color{incolor}26}]:} \PY{n}{a} \PY{o}{=} \PY{p}{(}\PY{l+m+mi}{1}\PY{p}{,}\PY{l+m+mi}{2}\PY{p}{,}\PY{l+m+mi}{3}\PY{p}{)} \PY{c+c1}{\PYZsh{}tuples use rounded parantheses for assignment}
         \PY{n}{t} \PY{o}{=} \PY{l+s+s1}{\PYZsq{}}\PY{l+s+s1}{a}\PY{l+s+s1}{\PYZsq{}}\PY{p}{,}\PY{l+s+s1}{\PYZsq{}}\PY{l+s+s1}{b}\PY{l+s+s1}{\PYZsq{}}\PY{p}{,}\PY{l+s+s1}{\PYZsq{}}\PY{l+s+s1}{c}\PY{l+s+s1}{\PYZsq{}} \PY{c+c1}{\PYZsh{}Paranthesis are not compulsory to define tuples, but using paranthesis is recommended}
         \PY{k}{print}\PY{p}{(}\PY{n+nb}{type}\PY{p}{(}\PY{n}{a}\PY{p}{)}\PY{p}{)}
         \PY{k}{print}\PY{p}{(}\PY{n}{a}\PY{p}{)}
         \PY{k}{print}\PY{p}{(}\PY{n+nb}{type}\PY{p}{(}\PY{n}{t}\PY{p}{)}\PY{p}{)}
         \PY{k}{print}\PY{p}{(}\PY{n}{t}\PY{p}{)}
\end{Verbatim}


    \begin{Verbatim}[commandchars=\\\{\}]
<type 'tuple'>
(1, 2, 3)
<type 'tuple'>
('a', 'b', 'c')

    \end{Verbatim}

    \subsubsection{Special Case for Tuples - Creating singleton or Empty
tuples}\label{special-case-for-tuples---creating-singleton-or-empty-tuples}

    "A special problem is the construction of tuples containing 0 or 1
items: the syntax has some extra quirks to accommodate these. Empty
tuples are constructed by an empty pair of parentheses; a tuple with one
item is constructed by following a value with a comma (it is not
sufficient to enclose a single value in parentheses). Ugly, but
effective."

    \begin{Verbatim}[commandchars=\\\{\}]
{\color{incolor}In [{\color{incolor}44}]:} \PY{n}{empty} \PY{o}{=} \PY{p}{(}\PY{p}{)}
         \PY{n}{singleton} \PY{o}{=} \PY{l+s+s1}{\PYZsq{}}\PY{l+s+s1}{hello}\PY{l+s+s1}{\PYZsq{}}\PY{p}{,}    \PY{c+c1}{\PYZsh{} \PYZlt{}\PYZhy{}\PYZhy{} note trailing comma}
         \PY{k}{print}\PY{p}{(}\PY{n+nb}{len}\PY{p}{(}\PY{n}{empty}\PY{p}{)}\PY{p}{)}
         \PY{k}{print}\PY{p}{(}\PY{n+nb}{type}\PY{p}{(}\PY{n}{empty}\PY{p}{)}\PY{p}{)}
         \PY{k}{print}
         \PY{k}{print}\PY{p}{(}\PY{n+nb}{len}\PY{p}{(}\PY{n}{singleton}\PY{p}{)}\PY{p}{)}
         \PY{k}{print}\PY{p}{(}\PY{n}{singleton}\PY{p}{)}
         \PY{k}{print}\PY{p}{(}\PY{n+nb}{type}\PY{p}{(}\PY{n}{singleton}\PY{p}{)}\PY{p}{)}
\end{Verbatim}


    \begin{Verbatim}[commandchars=\\\{\}]
0
<type 'tuple'>

1
('hello',)
<type 'tuple'>

    \end{Verbatim}

    Lists can store numbers characters, strings or mixed set of data

    \begin{Verbatim}[commandchars=\\\{\}]
{\color{incolor}In [{\color{incolor}21}]:} \PY{n}{a} \PY{o}{=} \PY{p}{[}\PY{l+s+s1}{\PYZsq{}}\PY{l+s+s1}{x}\PY{l+s+s1}{\PYZsq{}}\PY{p}{,}\PY{l+s+s1}{\PYZsq{}}\PY{l+s+s1}{y}\PY{l+s+s1}{\PYZsq{}}\PY{p}{,}\PY{l+s+s1}{\PYZsq{}}\PY{l+s+s1}{z}\PY{l+s+s1}{\PYZsq{}}\PY{p}{]} \PY{c+c1}{\PYZsh{}Lists can store any characters}
         \PY{n}{b} \PY{o}{=}\PY{p}{[}\PY{l+s+s1}{\PYZsq{}}\PY{l+s+s1}{abc}\PY{l+s+s1}{\PYZsq{}}\PY{p}{,}\PY{l+s+s1}{\PYZsq{}}\PY{l+s+s1}{cdf}\PY{l+s+s1}{\PYZsq{}}\PY{p}{,}\PY{l+s+s1}{\PYZsq{}}\PY{l+s+s1}{efg}\PY{l+s+s1}{\PYZsq{}}\PY{p}{]} \PY{c+c1}{\PYZsh{}lists can store strings}
         \PY{n}{c} \PY{o}{=} \PY{p}{[}\PY{l+m+mi}{1}\PY{p}{,} \PY{l+m+mi}{2}\PY{p}{,} \PY{l+m+mi}{5}\PY{o}{\PYZhy{}}\PY{l+m+mi}{1j}\PY{p}{,} \PY{l+s+s1}{\PYZsq{}}\PY{l+s+s1}{nepal}\PY{l+s+s1}{\PYZsq{}}\PY{p}{,}\PY{l+m+mi}{25}\PY{p}{,} \PY{l+s+s1}{\PYZsq{}}\PY{l+s+s1}{m}\PY{l+s+s1}{\PYZsq{}}\PY{p}{]} \PY{c+c1}{\PYZsh{}lists can store a mixed set of datatypes too}
         
         \PY{k}{print}\PY{p}{(}\PY{n}{a}\PY{p}{)}
         \PY{k}{print}\PY{p}{(}\PY{n}{b}\PY{p}{)}
         \PY{k}{print}\PY{p}{(}\PY{n}{c}\PY{p}{)}
\end{Verbatim}


    \begin{Verbatim}[commandchars=\\\{\}]
['x', 'y', 'z']
['abc', 'cdf', 'efg']
[1, 2, (5-1j), 'nepal', 25, 'm']

    \end{Verbatim}

    \begin{Verbatim}[commandchars=\\\{\}]
{\color{incolor}In [{\color{incolor}23}]:} \PY{n}{a} \PY{o}{=} \PY{p}{[}\PY{n}{x}\PY{p}{,}\PY{l+m+mi}{5}\PY{p}{,}\PY{n}{z}\PY{p}{]} \PY{c+c1}{\PYZsh{}characters must be defined inside quotation marks, otherwise it will be considered a variable}
         \PY{n}{a} \PY{c+c1}{\PYZsh{}since x has not been assigned earlier, this will throw an error}
\end{Verbatim}


    \begin{Verbatim}[commandchars=\\\{\}]

        

        NameErrorTraceback (most recent call last)

        <ipython-input-23-f8e48035200e> in <module>()
    ----> 1 a = [x,5,z] \#characters must be defined inside quotation marks, otherwise it will be considered a variable
          2 a \#since x has not been assigned earlier, this will throw an error
    

        NameError: name 'x' is not defined

    \end{Verbatim}

    Tuples can also store numbers, characters, strings or mixed set of data.

    \begin{Verbatim}[commandchars=\\\{\}]
{\color{incolor}In [{\color{incolor}22}]:} \PY{n}{a} \PY{o}{=} \PY{p}{(}\PY{l+s+s1}{\PYZsq{}}\PY{l+s+s1}{x}\PY{l+s+s1}{\PYZsq{}}\PY{p}{,}\PY{l+s+s1}{\PYZsq{}}\PY{l+s+s1}{y}\PY{l+s+s1}{\PYZsq{}}\PY{p}{,}\PY{l+s+s1}{\PYZsq{}}\PY{l+s+s1}{z}\PY{l+s+s1}{\PYZsq{}}\PY{p}{)} \PY{c+c1}{\PYZsh{}tuples can store any characters}
         \PY{n}{b} \PY{o}{=}\PY{p}{(}\PY{l+s+s1}{\PYZsq{}}\PY{l+s+s1}{abc}\PY{l+s+s1}{\PYZsq{}}\PY{p}{,}\PY{l+s+s1}{\PYZsq{}}\PY{l+s+s1}{cdf}\PY{l+s+s1}{\PYZsq{}}\PY{p}{,}\PY{l+s+s1}{\PYZsq{}}\PY{l+s+s1}{efg}\PY{l+s+s1}{\PYZsq{}}\PY{p}{)} \PY{c+c1}{\PYZsh{}tuples can store strings}
         \PY{n}{c} \PY{o}{=} \PY{p}{(}\PY{l+m+mi}{1}\PY{p}{,} \PY{l+m+mi}{2}\PY{p}{,} \PY{l+s+s1}{\PYZsq{}}\PY{l+s+s1}{x}\PY{l+s+s1}{\PYZsq{}}\PY{p}{,} \PY{l+m+mi}{1}\PY{o}{\PYZhy{}}\PY{l+m+mi}{2j}\PY{p}{)} \PY{c+c1}{\PYZsh{}tuples can store a mixed set of datatypes too}
         
         \PY{k}{print}\PY{p}{(}\PY{n}{a}\PY{p}{)}
         \PY{k}{print}\PY{p}{(}\PY{n}{b}\PY{p}{)}
         \PY{k}{print}\PY{p}{(}\PY{n}{c}\PY{p}{)}
\end{Verbatim}


    \begin{Verbatim}[commandchars=\\\{\}]
('x', 'y', 'z')
('abc', 'cdf', 'efg')
(1, 2, 'x', (1-2j))

    \end{Verbatim}

    \begin{Verbatim}[commandchars=\\\{\}]
{\color{incolor}In [{\color{incolor}24}]:} \PY{n}{a} \PY{o}{=} \PY{p}{(}\PY{n}{x}\PY{p}{,}\PY{l+m+mi}{5}\PY{p}{,}\PY{n}{z}\PY{p}{)} \PY{c+c1}{\PYZsh{}characters must be defined inside quotation marks, otherwise it will be considered a variable}
         \PY{n}{a} \PY{c+c1}{\PYZsh{}since x has not been assigned earlier, this will throw an error}
\end{Verbatim}


    \begin{Verbatim}[commandchars=\\\{\}]

        

        NameErrorTraceback (most recent call last)

        <ipython-input-24-6a86548650d9> in <module>()
    ----> 1 a = (x,5,z) \#characters must be defined inside quotation marks, otherwise it will be considered a variable
          2 a \#since x has not been assigned earlier, this will throw an error
    

        NameError: name 'x' is not defined

    \end{Verbatim}

    \subsection{Accessing items in Lists and
Tuples}\label{accessing-items-in-lists-and-tuples}

    You recall values from lists and tuples in exactly the same way.

It is very important to note that the indexing in Lists and Tuples start
from 0.

    \begin{Verbatim}[commandchars=\\\{\}]
{\color{incolor}In [{\color{incolor}29}]:} \PY{n}{cats} \PY{o}{=} \PY{p}{[}\PY{l+s+s1}{\PYZsq{}}\PY{l+s+s1}{Tom}\PY{l+s+s1}{\PYZsq{}}\PY{p}{,} \PY{l+s+s1}{\PYZsq{}}\PY{l+s+s1}{Snappy}\PY{l+s+s1}{\PYZsq{}}\PY{p}{,} \PY{l+s+s1}{\PYZsq{}}\PY{l+s+s1}{Kitty}\PY{l+s+s1}{\PYZsq{}}\PY{p}{,} \PY{l+s+s1}{\PYZsq{}}\PY{l+s+s1}{Jessie}\PY{l+s+s1}{\PYZsq{}}\PY{p}{,} \PY{l+s+s1}{\PYZsq{}}\PY{l+s+s1}{Chester}\PY{l+s+s1}{\PYZsq{}}\PY{p}{]}  \PY{c+c1}{\PYZsh{}list}
         \PY{n}{months} \PY{o}{=} \PY{p}{(}\PY{l+s+s1}{\PYZsq{}}\PY{l+s+s1}{January}\PY{l+s+s1}{\PYZsq{}}\PY{p}{,}\PY{l+s+s1}{\PYZsq{}}\PY{l+s+s1}{February}\PY{l+s+s1}{\PYZsq{}}\PY{p}{,}\PY{l+s+s1}{\PYZsq{}}\PY{l+s+s1}{March}\PY{l+s+s1}{\PYZsq{}}\PY{p}{,}\PY{l+s+s1}{\PYZsq{}}\PY{l+s+s1}{April}\PY{l+s+s1}{\PYZsq{}}\PY{p}{,}\PY{l+s+s1}{\PYZsq{}}\PY{l+s+s1}{May}\PY{l+s+s1}{\PYZsq{}}\PY{p}{,}\PY{l+s+s1}{\PYZsq{}}\PY{l+s+s1}{June}\PY{l+s+s1}{\PYZsq{}}\PY{p}{,}\PY{l+s+s1}{\PYZsq{}}\PY{l+s+s1}{July}\PY{l+s+s1}{\PYZsq{}}\PY{p}{,}\PY{l+s+s1}{\PYZsq{}}\PY{l+s+s1}{August}\PY{l+s+s1}{\PYZsq{}}\PY{p}{,}\PY{l+s+s1}{\PYZsq{}}\PY{l+s+s1}{September}\PY{l+s+s1}{\PYZsq{}}\PY{p}{,}\PY{l+s+s1}{\PYZsq{}}\PY{l+s+s1}{October}\PY{l+s+s1}{\PYZsq{}}\PY{p}{,}\PY{l+s+s1}{\PYZsq{}}\PY{l+s+s1}{November}\PY{l+s+s1}{\PYZsq{}}\PY{p}{,}\PY{l+s+s1}{\PYZsq{}}\PY{l+s+s1}{  December}\PY{l+s+s1}{\PYZsq{}}\PY{p}{)} \PY{c+c1}{\PYZsh{}tuple}
         
         \PY{k}{print}\PY{p}{(}\PY{l+s+s1}{\PYZsq{}}\PY{l+s+s1}{The third cat is }\PY{l+s+s1}{\PYZsq{}} \PY{o}{+} \PY{n}{cats}\PY{p}{[}\PY{l+m+mi}{2}\PY{p}{]}\PY{p}{)}
         \PY{k}{print}\PY{p}{(}\PY{l+s+s1}{\PYZsq{}}\PY{l+s+s1}{The third month is }\PY{l+s+s1}{\PYZsq{}}\PY{o}{+} \PY{n}{months}\PY{p}{[}\PY{l+m+mi}{2}\PY{p}{]}\PY{p}{)}
\end{Verbatim}


    \begin{Verbatim}[commandchars=\\\{\}]
The third cat is Kitty
The third month is March

    \end{Verbatim}

    It is possible to access elements of Lists and Tuples in a range. The
range can be defined is defined using square brackets for bot lists and
tuples.

Syntax: tuple\_or\_list\_name{[}start:end:step{]}

    \begin{Verbatim}[commandchars=\\\{\}]
{\color{incolor}In [{\color{incolor}35}]:} \PY{k}{print}\PY{p}{(}\PY{l+s+s1}{\PYZsq{}}\PY{l+s+s1}{The first 3 months are as follows:}\PY{l+s+s1}{\PYZsq{}}\PY{p}{)}
         \PY{k}{print}\PY{p}{(}\PY{n}{months}\PY{p}{[}\PY{l+m+mi}{0}\PY{p}{:}\PY{l+m+mi}{2}\PY{p}{:}\PY{l+m+mi}{1}\PY{p}{]}\PY{p}{)} \PY{c+c1}{\PYZsh{}months is a tuple here}
         
         \PY{k}{print}\PY{p}{(}\PY{l+s+s1}{\PYZsq{}}\PY{l+s+s1}{The first 3 cats are as follows:}\PY{l+s+s1}{\PYZsq{}}\PY{p}{)}
         \PY{k}{print}\PY{p}{(}\PY{n}{cats}\PY{p}{[}\PY{l+m+mi}{0}\PY{p}{:}\PY{l+m+mi}{2}\PY{p}{]}\PY{p}{)}\PY{c+c1}{\PYZsh{}if the step value is not mentioned, the default step value is considered as 1}
         
         \PY{k}{print}
         
         \PY{k}{print}\PY{p}{(}\PY{n}{months}\PY{p}{[}\PY{p}{:}\PY{p}{]}\PY{p}{)} 
         \PY{c+c1}{\PYZsh{}if the start value is not mentioned it refers to the begining of the list or the tuple}
         \PY{c+c1}{\PYZsh{}if the end value if not mentioned it refers to the end of the list or the tuple}
         \PY{c+c1}{\PYZsh{}if both are not mentioned it basically means the complete list or tuple}
\end{Verbatim}


    \begin{Verbatim}[commandchars=\\\{\}]
The first 3 months are as follows:
('January', 'February')
The first 3 cats are as follows:
['Tom', 'Snappy']

('January', 'February', 'March', 'April', 'May', 'June', 'July', 'August', 'September', 'October', 'November', '  December')

    \end{Verbatim}

    In the above examples, the 'slices' of lists and tuples that we have
obtained can be understood as the subsets of the lists or the tuples.
And the slices themselves are of the tupleType. Therefore, we can not
concatenate the them with stringTypes.

    \begin{Verbatim}[commandchars=\\\{\}]
{\color{incolor}In [{\color{incolor}43}]:} \PY{k}{print}\PY{p}{(}\PY{n}{months}\PY{p}{[}\PY{l+m+mi}{0}\PY{p}{:}\PY{l+m+mi}{11}\PY{p}{:}\PY{l+m+mi}{2}\PY{p}{]}\PY{p}{)}
         \PY{k}{print}\PY{p}{(}\PY{n+nb}{type}\PY{p}{(}\PY{n}{months}\PY{p}{[}\PY{l+m+mi}{0}\PY{p}{:}\PY{l+m+mi}{11}\PY{p}{:}\PY{l+m+mi}{2}\PY{p}{]}\PY{p}{)}\PY{p}{)}
         \PY{k}{print}  \PY{c+c1}{\PYZsh{}this line is for printing an empty line}
         \PY{k}{print}\PY{p}{(}\PY{n}{cats}\PY{p}{[}\PY{p}{:}\PY{l+m+mi}{3}\PY{p}{]}\PY{p}{)}
         \PY{k}{print}\PY{p}{(}\PY{n+nb}{type}\PY{p}{(}\PY{n}{cats}\PY{p}{[}\PY{p}{:}\PY{l+m+mi}{3}\PY{p}{]}\PY{p}{)}\PY{p}{)}
         \PY{k}{print}\PY{p}{(}\PY{l+s+s1}{\PYZsq{}}\PY{l+s+s1}{\PYZsq{}}\PY{p}{)} \PY{c+c1}{\PYZsh{}this line is also for printing an empty line}
         \PY{c+c1}{\PYZsh{}the following code will generate error as we are trying to concatenate stringType with TupleType}
         \PY{k}{print}\PY{p}{(}\PY{l+s+s1}{\PYZsq{}}\PY{l+s+s1}{The odd months are }\PY{l+s+s1}{\PYZsq{}}\PY{o}{+} \PY{n}{months}\PY{p}{[}\PY{l+m+mi}{0}\PY{p}{:}\PY{l+m+mi}{11}\PY{p}{:}\PY{l+m+mi}{2}\PY{p}{]}\PY{p}{)}
\end{Verbatim}


    \begin{Verbatim}[commandchars=\\\{\}]
('January', 'March', 'May', 'July', 'September', 'November')
<type 'tuple'>

['Tom', 'Snappy', 'Kitty']
<type 'list'>


    \end{Verbatim}

    \begin{Verbatim}[commandchars=\\\{\}]

        

        TypeErrorTraceback (most recent call last)

        <ipython-input-43-866a47ae44e3> in <module>()
          6 print('') \#this line is also for printing an empty line
          7 \#the following code will generate error as we are trying to concatenate stringType with TupleType
    ----> 8 print('The odd months are '+ months[0:11:2])
    

        TypeError: cannot concatenate 'str' and 'tuple' objects

    \end{Verbatim}

    \subsection{Actions on Lists and
Tuples}\label{actions-on-lists-and-tuples}

    \subparagraph{The major difference between Tuples and Lists is that
Tuples are immutable while Lists are
mutable.}\label{the-major-difference-between-tuples-and-lists-is-that-tuples-are-immutable-while-lists-are-mutable.}

    This basically means that the elements in Tuples cannot be changed. And
thus the elements in tuples CANNOT be updated or removed. Hence, there
are no actions that are associated with the data type Tuples.

However the tuple type objects can call two functions - COUNT and INDEX.

    \begin{Verbatim}[commandchars=\\\{\}]
{\color{incolor}In [{\color{incolor}47}]:} \PY{n}{vowels} \PY{o}{=} \PY{p}{(}\PY{l+s+s1}{\PYZsq{}}\PY{l+s+s1}{a}\PY{l+s+s1}{\PYZsq{}}\PY{p}{,} \PY{l+s+s1}{\PYZsq{}}\PY{l+s+s1}{e}\PY{l+s+s1}{\PYZsq{}}\PY{p}{,} \PY{l+s+s1}{\PYZsq{}}\PY{l+s+s1}{i}\PY{l+s+s1}{\PYZsq{}}\PY{p}{,} \PY{l+s+s1}{\PYZsq{}}\PY{l+s+s1}{o}\PY{l+s+s1}{\PYZsq{}}\PY{p}{,} \PY{l+s+s1}{\PYZsq{}}\PY{l+s+s1}{i}\PY{l+s+s1}{\PYZsq{}}\PY{p}{,} \PY{l+s+s1}{\PYZsq{}}\PY{l+s+s1}{o}\PY{l+s+s1}{\PYZsq{}}\PY{p}{,} \PY{l+s+s1}{\PYZsq{}}\PY{l+s+s1}{e}\PY{l+s+s1}{\PYZsq{}}\PY{p}{,} \PY{l+s+s1}{\PYZsq{}}\PY{l+s+s1}{i}\PY{l+s+s1}{\PYZsq{}}\PY{p}{,} \PY{l+s+s1}{\PYZsq{}}\PY{l+s+s1}{u}\PY{l+s+s1}{\PYZsq{}}\PY{p}{)} \PY{c+c1}{\PYZsh{} vowels tuple}
         \PY{n}{count} \PY{o}{=} \PY{n}{vowels}\PY{o}{.}\PY{n}{count}\PY{p}{(}\PY{l+s+s1}{\PYZsq{}}\PY{l+s+s1}{i}\PY{l+s+s1}{\PYZsq{}}\PY{p}{)}  \PY{c+c1}{\PYZsh{} count element \PYZsq{}i\PYZsq{}}
         \PY{k}{print}\PY{p}{(}\PY{l+s+s1}{\PYZsq{}}\PY{l+s+s1}{i occurs the following times in the tuple: }\PY{l+s+s1}{\PYZsq{}}\PY{p}{,} \PY{n}{count}\PY{p}{)}
         \PY{k}{print}
         \PY{n}{count} \PY{o}{=} \PY{n}{vowels}\PY{o}{.}\PY{n}{count}\PY{p}{(}\PY{l+s+s1}{\PYZsq{}}\PY{l+s+s1}{p}\PY{l+s+s1}{\PYZsq{}}\PY{p}{)} \PY{c+c1}{\PYZsh{} \PYZsh{} count element \PYZsq{}p\PYZsq{}}
         \PY{k}{print}\PY{p}{(}\PY{l+s+s1}{\PYZsq{}}\PY{l+s+s1}{p occurs the following times in the tuple: }\PY{l+s+s1}{\PYZsq{}}\PY{p}{,} \PY{n}{count}\PY{p}{)}
\end{Verbatim}


    \begin{Verbatim}[commandchars=\\\{\}]
('i occurs the following times in the tuple: ', 3)

('p occurs the following times in the tuple: ', 0)

    \end{Verbatim}

    \begin{Verbatim}[commandchars=\\\{\}]
{\color{incolor}In [{\color{incolor}48}]:} \PY{n}{index} \PY{o}{=} \PY{n}{vowels}\PY{o}{.}\PY{n}{index}\PY{p}{(}\PY{l+s+s1}{\PYZsq{}}\PY{l+s+s1}{e}\PY{l+s+s1}{\PYZsq{}}\PY{p}{)} \PY{c+c1}{\PYZsh{}indexing element e}
         \PY{k}{print}\PY{p}{(}\PY{l+s+s1}{\PYZsq{}}\PY{l+s+s1}{The index of e:}\PY{l+s+s1}{\PYZsq{}}\PY{p}{,} \PY{n}{index}\PY{p}{)}
         
         \PY{k}{print}
         
         \PY{n}{index} \PY{o}{=} \PY{n}{vowels}\PY{o}{.}\PY{n}{index}\PY{p}{(}\PY{l+s+s1}{\PYZsq{}}\PY{l+s+s1}{i}\PY{l+s+s1}{\PYZsq{}}\PY{p}{)} \PY{c+c1}{\PYZsh{}indexing element i}
         \PY{k}{print}\PY{p}{(}\PY{l+s+s1}{\PYZsq{}}\PY{l+s+s1}{The index of i:}\PY{l+s+s1}{\PYZsq{}}\PY{p}{,} \PY{n}{index}\PY{p}{)}
         \PY{c+c1}{\PYZsh{} only the first index of the element is printed}
\end{Verbatim}


    \begin{Verbatim}[commandchars=\\\{\}]
('The index of e:', 1)

('The index of i:', 2)

    \end{Verbatim}

    It is also not possible to delete or remove an element from a tuple.
However, del statement can be used to remove or delete an entire tuple.

    \begin{Verbatim}[commandchars=\\\{\}]
{\color{incolor}In [{\color{incolor}50}]:} \PY{n}{tup} \PY{o}{=} \PY{p}{(}\PY{l+s+s1}{\PYZsq{}}\PY{l+s+s1}{physics}\PY{l+s+s1}{\PYZsq{}}\PY{p}{,} \PY{l+s+s1}{\PYZsq{}}\PY{l+s+s1}{chemistry}\PY{l+s+s1}{\PYZsq{}}\PY{p}{,} \PY{l+m+mi}{1997}\PY{p}{,} \PY{l+m+mi}{2000}\PY{p}{)}
         \PY{k}{print} \PY{n}{tup}
         \PY{k}{del} \PY{n}{tup}
         \PY{k}{print} \PY{l+s+s2}{\PYZdq{}}\PY{l+s+s2}{After deleting tup : }\PY{l+s+s2}{\PYZdq{}}
         \PY{k}{print} \PY{n}{tup} \PY{c+c1}{\PYZsh{}throws an error because no such tuple exists}
\end{Verbatim}


    \begin{Verbatim}[commandchars=\\\{\}]
('physics', 'chemistry', 1997, 2000)
After deleting tup : 

    \end{Verbatim}

    \begin{Verbatim}[commandchars=\\\{\}]

        

        NameErrorTraceback (most recent call last)

        <ipython-input-50-d77d02d08d83> in <module>()
          3 del tup
          4 print "After deleting tup : "
    ----> 5 print tup \#throws an error because no such tuple exists
    

        NameError: name 'tup' is not defined

    \end{Verbatim}

    \subsubsection{Actions on List Type
Objects}\label{actions-on-list-type-objects}

    As mentioned earlier, Lists are mutable. This means that the elements of
a List type object can be changed or elements can be added and deleted.

    \subparagraph{Append() and Insert()}\label{append-and-insert}

The append() method adds a single item to the existing list. It doesn't
return a new list; rather it modifies the original list. You can also
add

The syntax of append() method is:

list.append(item)

    \begin{Verbatim}[commandchars=\\\{\}]
{\color{incolor}In [{\color{incolor}52}]:} \PY{n}{animal} \PY{o}{=} \PY{p}{[}\PY{l+s+s1}{\PYZsq{}}\PY{l+s+s1}{cat}\PY{l+s+s1}{\PYZsq{}}\PY{p}{,} \PY{l+s+s1}{\PYZsq{}}\PY{l+s+s1}{dog}\PY{l+s+s1}{\PYZsq{}}\PY{p}{,} \PY{l+s+s1}{\PYZsq{}}\PY{l+s+s1}{rabbit}\PY{l+s+s1}{\PYZsq{}}\PY{p}{]}
         
         \PY{n}{animal}\PY{o}{.}\PY{n}{append}\PY{p}{(}\PY{l+s+s1}{\PYZsq{}}\PY{l+s+s1}{tiger}\PY{l+s+s1}{\PYZsq{}}\PY{p}{)} \PY{c+c1}{\PYZsh{} adding new element to the list}
         
         \PY{k}{print}\PY{p}{(}\PY{l+s+s1}{\PYZsq{}}\PY{l+s+s1}{Updated animal list: }\PY{l+s+s1}{\PYZsq{}}\PY{p}{,} \PY{n}{animal}\PY{p}{)} \PY{c+c1}{\PYZsh{}Updated animal List}
\end{Verbatim}


    \begin{Verbatim}[commandchars=\\\{\}]
('Updated animal list: ', ['cat', 'dog', 'rabbit', 'tiger'])

    \end{Verbatim}

    \begin{Verbatim}[commandchars=\\\{\}]
{\color{incolor}In [{\color{incolor}24}]:} \PY{n}{animal} \PY{o}{=} \PY{p}{[}\PY{l+s+s1}{\PYZsq{}}\PY{l+s+s1}{cat}\PY{l+s+s1}{\PYZsq{}}\PY{p}{,} \PY{l+s+s1}{\PYZsq{}}\PY{l+s+s1}{dog}\PY{l+s+s1}{\PYZsq{}}\PY{p}{,} \PY{l+s+s1}{\PYZsq{}}\PY{l+s+s1}{rabbit}\PY{l+s+s1}{\PYZsq{}}\PY{p}{]}
         \PY{n}{wild\PYZus{}animal} \PY{o}{=} \PY{p}{[}\PY{l+s+s1}{\PYZsq{}}\PY{l+s+s1}{tiger}\PY{l+s+s1}{\PYZsq{}}\PY{p}{,} \PY{l+s+s1}{\PYZsq{}}\PY{l+s+s1}{fox}\PY{l+s+s1}{\PYZsq{}}\PY{p}{]}
         
         \PY{n}{animal}\PY{o}{.}\PY{n}{append}\PY{p}{(}\PY{n}{wild\PYZus{}animal}\PY{p}{)} \PY{c+c1}{\PYZsh{} adding wild\PYZus{}animal list to animal list}
         
         \PY{k}{print}\PY{p}{(}\PY{l+s+s1}{\PYZsq{}}\PY{l+s+s1}{Updated animal list: }\PY{l+s+s1}{\PYZsq{}}\PY{p}{,} \PY{n}{animal}\PY{p}{)}
         \PY{k}{print}
         \PY{k}{print}\PY{p}{(}\PY{l+s+s1}{\PYZsq{}}\PY{l+s+s1}{It is important to note that the list wild\PYZus{}animal is added to the animal list as a single element.}\PY{l+s+s1}{\PYZsq{}}\PY{p}{)}
         \PY{k}{print}\PY{p}{(}\PY{l+s+s1}{\PYZsq{}}\PY{l+s+s1}{Only one element of type List has been added to the existing list}\PY{l+s+s1}{\PYZsq{}}\PY{p}{)}
\end{Verbatim}


    \begin{Verbatim}[commandchars=\\\{\}]
('Updated animal list: ', ['cat', 'dog', 'rabbit', ['tiger', 'fox']])

It is important to note that the list wild\_animal is added to the animal list as a single element.
Only one element of type List has been added to the existing list

    \end{Verbatim}

    We can insert items into lists using insert function.

    \begin{Verbatim}[commandchars=\\\{\}]
{\color{incolor}In [{\color{incolor}27}]:} \PY{n}{animal}\PY{o}{.}\PY{n}{insert}\PY{p}{(}\PY{l+m+mi}{2}\PY{p}{,}\PY{l+s+s1}{\PYZsq{}}\PY{l+s+s1}{wildcat}\PY{l+s+s1}{\PYZsq{}}\PY{p}{)} \PY{c+c1}{\PYZsh{}inserts wildcat at index 2}
         \PY{k}{print}\PY{p}{(}\PY{l+s+s1}{\PYZsq{}}\PY{l+s+s1}{Updated animal list: }\PY{l+s+s1}{\PYZsq{}}\PY{p}{,} \PY{n}{animal}\PY{p}{)}
\end{Verbatim}


    \begin{Verbatim}[commandchars=\\\{\}]
('Updated animal list: ', ['cat', 'dog', 'wildcat', 'wildcat', 'rabbit', ['tiger', 'fox']])

    \end{Verbatim}

    \subparagraph{Extend()}\label{extend}

If in the above case, we want to add the elements of the list
wild\_animal to the existing animal list, then we have to use the
extend() function. The extend() extends the list by adding all items of
a list (passed as an argument) to the end. We can also add elements of a
tuple or a set data type to an existing list.

Syntax: list1.extend(list2)

    \begin{Verbatim}[commandchars=\\\{\}]
{\color{incolor}In [{\color{incolor}58}]:} \PY{c+c1}{\PYZsh{} language list}
         \PY{n}{language} \PY{o}{=} \PY{p}{[}\PY{l+s+s1}{\PYZsq{}}\PY{l+s+s1}{French}\PY{l+s+s1}{\PYZsq{}}\PY{p}{,} \PY{l+s+s1}{\PYZsq{}}\PY{l+s+s1}{English}\PY{l+s+s1}{\PYZsq{}}\PY{p}{,} \PY{l+s+s1}{\PYZsq{}}\PY{l+s+s1}{German}\PY{l+s+s1}{\PYZsq{}}\PY{p}{]}
         
         \PY{c+c1}{\PYZsh{} another list of language}
         \PY{n}{language1} \PY{o}{=} \PY{p}{[}\PY{l+s+s1}{\PYZsq{}}\PY{l+s+s1}{Spanish}\PY{l+s+s1}{\PYZsq{}}\PY{p}{,} \PY{l+s+s1}{\PYZsq{}}\PY{l+s+s1}{Portuguese}\PY{l+s+s1}{\PYZsq{}}\PY{p}{]}
         
         \PY{n}{language}\PY{o}{.}\PY{n}{extend}\PY{p}{(}\PY{n}{language1}\PY{p}{)}
         
         \PY{c+c1}{\PYZsh{} Extended List}
         \PY{k}{print}\PY{p}{(}\PY{l+s+s1}{\PYZsq{}}\PY{l+s+s1}{Language List: }\PY{l+s+s1}{\PYZsq{}}\PY{p}{,} \PY{n}{language}\PY{p}{)}
\end{Verbatim}


    \begin{Verbatim}[commandchars=\\\{\}]
('Language List: ', ['French', 'English', 'German', 'Spanish', 'Portuguese'])

    \end{Verbatim}

    \begin{Verbatim}[commandchars=\\\{\}]
{\color{incolor}In [{\color{incolor}61}]:} \PY{c+c1}{\PYZsh{} language list}
         \PY{n}{language} \PY{o}{=} \PY{p}{[}\PY{l+s+s1}{\PYZsq{}}\PY{l+s+s1}{French}\PY{l+s+s1}{\PYZsq{}}\PY{p}{,} \PY{l+s+s1}{\PYZsq{}}\PY{l+s+s1}{English}\PY{l+s+s1}{\PYZsq{}}\PY{p}{,} \PY{l+s+s1}{\PYZsq{}}\PY{l+s+s1}{German}\PY{l+s+s1}{\PYZsq{}}\PY{p}{]}
         
         \PY{c+c1}{\PYZsh{} language tuple}
         \PY{n}{language\PYZus{}tuple} \PY{o}{=} \PY{p}{(}\PY{l+s+s1}{\PYZsq{}}\PY{l+s+s1}{Spanish}\PY{l+s+s1}{\PYZsq{}}\PY{p}{,} \PY{l+s+s1}{\PYZsq{}}\PY{l+s+s1}{Portuguese}\PY{l+s+s1}{\PYZsq{}}\PY{p}{)}
         
         \PY{c+c1}{\PYZsh{} appending element of a tuple to an existing list}
         \PY{n}{language}\PY{o}{.}\PY{n}{extend}\PY{p}{(}\PY{n}{language\PYZus{}tuple}\PY{p}{)}
         
         \PY{k}{print}\PY{p}{(}\PY{l+s+s1}{\PYZsq{}}\PY{l+s+s1}{New Language List: }\PY{l+s+s1}{\PYZsq{}}\PY{p}{,} \PY{n}{language}\PY{p}{)}
\end{Verbatim}


    \begin{Verbatim}[commandchars=\\\{\}]
('New Language List: ', ['French', 'English', 'German', 'Spanish', 'Portuguese'])

    \end{Verbatim}

    We can also add the elements of a list to anthor list by using the
operator +=

    \begin{Verbatim}[commandchars=\\\{\}]
{\color{incolor}In [{\color{incolor}63}]:} \PY{n}{a} \PY{o}{=} \PY{p}{[}\PY{l+m+mi}{1}\PY{p}{,} \PY{l+m+mi}{2}\PY{p}{]}
         \PY{n}{b} \PY{o}{=} \PY{p}{[}\PY{l+m+mi}{3}\PY{p}{,} \PY{l+m+mi}{4}\PY{p}{]}
         
         \PY{n}{a} \PY{o}{+}\PY{o}{=} \PY{n}{b}
         \PY{k}{print}\PY{p}{(}\PY{n}{a}\PY{p}{)}
\end{Verbatim}


    \begin{Verbatim}[commandchars=\\\{\}]
[1, 2, 3, 4]

    \end{Verbatim}

    \subparagraph{Insert()}\label{insert}

The insert() method inserts the element to the list at the given index.

Syntax: list.insert(index, element)

    \begin{Verbatim}[commandchars=\\\{\}]
{\color{incolor}In [{\color{incolor}64}]:} \PY{c+c1}{\PYZsh{} vowel list}
         \PY{n}{vowel} \PY{o}{=} \PY{p}{[}\PY{l+s+s1}{\PYZsq{}}\PY{l+s+s1}{a}\PY{l+s+s1}{\PYZsq{}}\PY{p}{,} \PY{l+s+s1}{\PYZsq{}}\PY{l+s+s1}{e}\PY{l+s+s1}{\PYZsq{}}\PY{p}{,} \PY{l+s+s1}{\PYZsq{}}\PY{l+s+s1}{i}\PY{l+s+s1}{\PYZsq{}}\PY{p}{,} \PY{l+s+s1}{\PYZsq{}}\PY{l+s+s1}{u}\PY{l+s+s1}{\PYZsq{}}\PY{p}{]}
         
         \PY{c+c1}{\PYZsh{} inserting element to list at 4th position}
         \PY{n}{vowel}\PY{o}{.}\PY{n}{insert}\PY{p}{(}\PY{l+m+mi}{3}\PY{p}{,} \PY{l+s+s1}{\PYZsq{}}\PY{l+s+s1}{o}\PY{l+s+s1}{\PYZsq{}}\PY{p}{)}
         
         \PY{k}{print}\PY{p}{(}\PY{l+s+s1}{\PYZsq{}}\PY{l+s+s1}{Updated List: }\PY{l+s+s1}{\PYZsq{}}\PY{p}{,} \PY{n}{vowel}\PY{p}{)}
\end{Verbatim}


    \begin{Verbatim}[commandchars=\\\{\}]
('Updated List: ', ['a', 'e', 'i', 'o', 'u'])

    \end{Verbatim}

    \subparagraph{Remove()}\label{remove}

The remove() method searches for the given element in the list and
removes the first matching element.

Sytax: list.remove(element)

    \begin{Verbatim}[commandchars=\\\{\}]
{\color{incolor}In [{\color{incolor}67}]:} \PY{n}{animal} \PY{o}{=} \PY{p}{[}\PY{l+s+s1}{\PYZsq{}}\PY{l+s+s1}{cat}\PY{l+s+s1}{\PYZsq{}}\PY{p}{,} \PY{l+s+s1}{\PYZsq{}}\PY{l+s+s1}{dog}\PY{l+s+s1}{\PYZsq{}}\PY{p}{,} \PY{l+s+s1}{\PYZsq{}}\PY{l+s+s1}{rabbit}\PY{l+s+s1}{\PYZsq{}}\PY{p}{,} \PY{l+s+s1}{\PYZsq{}}\PY{l+s+s1}{guinea pig}\PY{l+s+s1}{\PYZsq{}}\PY{p}{]}
         
         \PY{c+c1}{\PYZsh{} \PYZsq{}rabbit\PYZsq{} element is removed}
         \PY{n}{animal}\PY{o}{.}\PY{n}{remove}\PY{p}{(}\PY{l+s+s1}{\PYZsq{}}\PY{l+s+s1}{rabbit}\PY{l+s+s1}{\PYZsq{}}\PY{p}{)}
         
         \PY{c+c1}{\PYZsh{}Updated Animal List}
         \PY{k}{print}\PY{p}{(}\PY{l+s+s1}{\PYZsq{}}\PY{l+s+s1}{Updated animal list: }\PY{l+s+s1}{\PYZsq{}}\PY{p}{,} \PY{n}{animal}\PY{p}{)}
         
         \PY{k}{print}
         \PY{n}{animal1} \PY{o}{=} \PY{p}{[}\PY{l+s+s1}{\PYZsq{}}\PY{l+s+s1}{cat}\PY{l+s+s1}{\PYZsq{}}\PY{p}{,}\PY{l+s+s1}{\PYZsq{}}\PY{l+s+s1}{dog}\PY{l+s+s1}{\PYZsq{}}\PY{p}{,}\PY{l+s+s1}{\PYZsq{}}\PY{l+s+s1}{fox}\PY{l+s+s1}{\PYZsq{}}\PY{p}{,}\PY{l+s+s1}{\PYZsq{}}\PY{l+s+s1}{dog}\PY{l+s+s1}{\PYZsq{}}\PY{p}{,}\PY{l+s+s1}{\PYZsq{}}\PY{l+s+s1}{dog}\PY{l+s+s1}{\PYZsq{}}\PY{p}{,}\PY{l+s+s1}{\PYZsq{}}\PY{l+s+s1}{cat}\PY{l+s+s1}{\PYZsq{}}\PY{p}{]}
         \PY{n}{animal1}\PY{o}{.}\PY{n}{remove}\PY{p}{(}\PY{l+s+s1}{\PYZsq{}}\PY{l+s+s1}{dog}\PY{l+s+s1}{\PYZsq{}}\PY{p}{)} \PY{c+c1}{\PYZsh{}removing repititive elements from a list will remove only one of them}
         \PY{k}{print}\PY{p}{(}\PY{l+s+s1}{\PYZsq{}}\PY{l+s+s1}{Updated animla1 list: }\PY{l+s+s1}{\PYZsq{}}\PY{p}{,}\PY{n}{animal1}\PY{p}{)}
         
         \PY{n}{animal}\PY{o}{.}\PY{n}{remove}\PY{p}{(}\PY{l+s+s1}{\PYZsq{}}\PY{l+s+s1}{wildcat}\PY{l+s+s1}{\PYZsq{}}\PY{p}{)} \PY{c+c1}{\PYZsh{}removing non existing element will cause error}
         \PY{k}{print}\PY{p}{(}\PY{l+s+s1}{\PYZsq{}}\PY{l+s+s1}{Updated animal list: }\PY{l+s+s1}{\PYZsq{}}\PY{p}{,}\PY{n}{animal}\PY{p}{)}
\end{Verbatim}


    \begin{Verbatim}[commandchars=\\\{\}]
('Updated animal list: ', ['cat', 'dog', 'guinea pig'])

('Update animla1 list: ', ['cat', 'fox', 'dog', 'dog', 'cat'])

    \end{Verbatim}

    \begin{Verbatim}[commandchars=\\\{\}]

        

        ValueErrorTraceback (most recent call last)

        <ipython-input-67-c7a3b9d93172> in <module>()
         12 print('Update animla1 list: ',animal1)
         13 
    ---> 14 animal.remove('wildcat') \#removing non existing element will cause error
         15 print('Updated animal list: ',animal)
         16 
    

        ValueError: list.remove(x): x not in list

    \end{Verbatim}

    \subparagraph{Pop()}\label{pop}

The pop() method removes and returns the element at the given index
(passed as an argument) from the list.

Syntax: list.pop(index)

    \begin{Verbatim}[commandchars=\\\{\}]
{\color{incolor}In [{\color{incolor}68}]:} \PY{n}{language} \PY{o}{=} \PY{p}{[}\PY{l+s+s1}{\PYZsq{}}\PY{l+s+s1}{Python}\PY{l+s+s1}{\PYZsq{}}\PY{p}{,} \PY{l+s+s1}{\PYZsq{}}\PY{l+s+s1}{Java}\PY{l+s+s1}{\PYZsq{}}\PY{p}{,} \PY{l+s+s1}{\PYZsq{}}\PY{l+s+s1}{C++}\PY{l+s+s1}{\PYZsq{}}\PY{p}{,} \PY{l+s+s1}{\PYZsq{}}\PY{l+s+s1}{French}\PY{l+s+s1}{\PYZsq{}}\PY{p}{,} \PY{l+s+s1}{\PYZsq{}}\PY{l+s+s1}{C}\PY{l+s+s1}{\PYZsq{}}\PY{p}{]}
         
         \PY{c+c1}{\PYZsh{} Return value from pop()}
         \PY{c+c1}{\PYZsh{} When 3 is passed}
         \PY{n}{return\PYZus{}value} \PY{o}{=} \PY{n}{language}\PY{o}{.}\PY{n}{pop}\PY{p}{(}\PY{l+m+mi}{3}\PY{p}{)}
         \PY{k}{print}\PY{p}{(}\PY{l+s+s1}{\PYZsq{}}\PY{l+s+s1}{Return Value: }\PY{l+s+s1}{\PYZsq{}}\PY{p}{,} \PY{n}{return\PYZus{}value}\PY{p}{)}
         
         \PY{c+c1}{\PYZsh{} Updated List}
         \PY{k}{print}\PY{p}{(}\PY{l+s+s1}{\PYZsq{}}\PY{l+s+s1}{Updated List: }\PY{l+s+s1}{\PYZsq{}}\PY{p}{,} \PY{n}{language}\PY{p}{)}
\end{Verbatim}


    \begin{Verbatim}[commandchars=\\\{\}]
('Return Value: ', 'French')
('Updated List: ', ['Python', 'Java', 'C++', 'C'])

    \end{Verbatim}

    \begin{Verbatim}[commandchars=\\\{\}]
{\color{incolor}In [{\color{incolor}69}]:} \PY{c+c1}{\PYZsh{} programming language list}
         \PY{n}{language} \PY{o}{=} \PY{p}{[}\PY{l+s+s1}{\PYZsq{}}\PY{l+s+s1}{Python}\PY{l+s+s1}{\PYZsq{}}\PY{p}{,} \PY{l+s+s1}{\PYZsq{}}\PY{l+s+s1}{Java}\PY{l+s+s1}{\PYZsq{}}\PY{p}{,} \PY{l+s+s1}{\PYZsq{}}\PY{l+s+s1}{C++}\PY{l+s+s1}{\PYZsq{}}\PY{p}{,} \PY{l+s+s1}{\PYZsq{}}\PY{l+s+s1}{Ruby}\PY{l+s+s1}{\PYZsq{}}\PY{p}{,} \PY{l+s+s1}{\PYZsq{}}\PY{l+s+s1}{C}\PY{l+s+s1}{\PYZsq{}}\PY{p}{]}
         
         \PY{c+c1}{\PYZsh{} When index is not passed}
         \PY{k}{print}\PY{p}{(}\PY{l+s+s1}{\PYZsq{}}\PY{l+s+s1}{When index is not passed:}\PY{l+s+s1}{\PYZsq{}}\PY{p}{)} 
         \PY{k}{print}\PY{p}{(}\PY{l+s+s1}{\PYZsq{}}\PY{l+s+s1}{Return Value: }\PY{l+s+s1}{\PYZsq{}}\PY{p}{,} \PY{n}{language}\PY{o}{.}\PY{n}{pop}\PY{p}{(}\PY{p}{)}\PY{p}{)}
         \PY{k}{print}\PY{p}{(}\PY{l+s+s1}{\PYZsq{}}\PY{l+s+s1}{Updated List: }\PY{l+s+s1}{\PYZsq{}}\PY{p}{,} \PY{n}{language}\PY{p}{)}
         
         \PY{c+c1}{\PYZsh{} When \PYZhy{}1 is passed}
         \PY{c+c1}{\PYZsh{} Pops Last Element}
         \PY{k}{print}\PY{p}{(}\PY{l+s+s1}{\PYZsq{}}\PY{l+s+se}{\PYZbs{}n}\PY{l+s+s1}{When \PYZhy{}1 is passed:}\PY{l+s+s1}{\PYZsq{}}\PY{p}{)} 
         \PY{k}{print}\PY{p}{(}\PY{l+s+s1}{\PYZsq{}}\PY{l+s+s1}{Return Value: }\PY{l+s+s1}{\PYZsq{}}\PY{p}{,} \PY{n}{language}\PY{o}{.}\PY{n}{pop}\PY{p}{(}\PY{o}{\PYZhy{}}\PY{l+m+mi}{1}\PY{p}{)}\PY{p}{)}
         \PY{k}{print}\PY{p}{(}\PY{l+s+s1}{\PYZsq{}}\PY{l+s+s1}{Updated List: }\PY{l+s+s1}{\PYZsq{}}\PY{p}{,} \PY{n}{language}\PY{p}{)}
         
         \PY{c+c1}{\PYZsh{} When \PYZhy{}3 is passed}
         \PY{c+c1}{\PYZsh{} Pops Third Last Element}
         \PY{k}{print}\PY{p}{(}\PY{l+s+s1}{\PYZsq{}}\PY{l+s+se}{\PYZbs{}n}\PY{l+s+s1}{When \PYZhy{}3 is passed:}\PY{l+s+s1}{\PYZsq{}}\PY{p}{)} 
         \PY{k}{print}\PY{p}{(}\PY{l+s+s1}{\PYZsq{}}\PY{l+s+s1}{Return Value: }\PY{l+s+s1}{\PYZsq{}}\PY{p}{,} \PY{n}{language}\PY{o}{.}\PY{n}{pop}\PY{p}{(}\PY{o}{\PYZhy{}}\PY{l+m+mi}{3}\PY{p}{)}\PY{p}{)}
         \PY{k}{print}\PY{p}{(}\PY{l+s+s1}{\PYZsq{}}\PY{l+s+s1}{Updated List: }\PY{l+s+s1}{\PYZsq{}}\PY{p}{,} \PY{n}{language}\PY{p}{)}
\end{Verbatim}


    \begin{Verbatim}[commandchars=\\\{\}]
When index is not passed:
('Return Value: ', 'C')
('Updated List: ', ['Python', 'Java', 'C++', 'Ruby'])

When -1 is passed:
('Return Value: ', 'Ruby')
('Updated List: ', ['Python', 'Java', 'C++'])

When -3 is passed:
('Return Value: ', 'Python')
('Updated List: ', ['Java', 'C++'])

    \end{Verbatim}

    \subparagraph{Reverse()}\label{reverse}

The reverse() method reverses the elements of a given list.

Syntax: list.reverse()

    \begin{Verbatim}[commandchars=\\\{\}]
{\color{incolor}In [{\color{incolor}70}]:} \PY{n}{os} \PY{o}{=} \PY{p}{[}\PY{l+s+s1}{\PYZsq{}}\PY{l+s+s1}{Windows}\PY{l+s+s1}{\PYZsq{}}\PY{p}{,} \PY{l+s+s1}{\PYZsq{}}\PY{l+s+s1}{macOS}\PY{l+s+s1}{\PYZsq{}}\PY{p}{,} \PY{l+s+s1}{\PYZsq{}}\PY{l+s+s1}{Linux}\PY{l+s+s1}{\PYZsq{}}\PY{p}{]}
         \PY{k}{print}\PY{p}{(}\PY{l+s+s1}{\PYZsq{}}\PY{l+s+s1}{Original List:}\PY{l+s+s1}{\PYZsq{}}\PY{p}{,} \PY{n}{os}\PY{p}{)}
         
         \PY{c+c1}{\PYZsh{} List Reverse}
         \PY{n}{os}\PY{o}{.}\PY{n}{reverse}\PY{p}{(}\PY{p}{)}
         
         \PY{c+c1}{\PYZsh{} updated list}
         \PY{k}{print}\PY{p}{(}\PY{l+s+s1}{\PYZsq{}}\PY{l+s+s1}{Updated List:}\PY{l+s+s1}{\PYZsq{}}\PY{p}{,} \PY{n}{os}\PY{p}{)}
\end{Verbatim}


    \begin{Verbatim}[commandchars=\\\{\}]
('Original List:', ['Windows', 'macOS', 'Linux'])
('Updated List:', ['Linux', 'macOS', 'Windows'])

    \end{Verbatim}

    Reversing a lis using SLICING operators

Syntax: reversed\_list = os{[}start:stop:step{]}

    \begin{Verbatim}[commandchars=\\\{\}]
{\color{incolor}In [{\color{incolor}73}]:} \PY{n}{os} \PY{o}{=} \PY{p}{[}\PY{l+s+s1}{\PYZsq{}}\PY{l+s+s1}{Windows}\PY{l+s+s1}{\PYZsq{}}\PY{p}{,} \PY{l+s+s1}{\PYZsq{}}\PY{l+s+s1}{macOS}\PY{l+s+s1}{\PYZsq{}}\PY{p}{,} \PY{l+s+s1}{\PYZsq{}}\PY{l+s+s1}{Linux}\PY{l+s+s1}{\PYZsq{}}\PY{p}{]}
         \PY{k}{print}\PY{p}{(}\PY{l+s+s1}{\PYZsq{}}\PY{l+s+s1}{Original List:}\PY{l+s+s1}{\PYZsq{}}\PY{p}{,} \PY{n}{os}\PY{p}{)}
         
         \PY{n}{reversed\PYZus{}list} \PY{o}{=} \PY{n}{os}\PY{p}{[}\PY{p}{:}\PY{p}{:}\PY{o}{\PYZhy{}}\PY{l+m+mi}{1}\PY{p}{]}
         
         \PY{k}{print}\PY{p}{(}\PY{l+s+s1}{\PYZsq{}}\PY{l+s+s1}{Updated List:}\PY{l+s+s1}{\PYZsq{}}\PY{p}{,} \PY{n}{reversed\PYZus{}list}\PY{p}{)}
\end{Verbatim}


    \begin{Verbatim}[commandchars=\\\{\}]
('Original List:', ['Windows', 'macOS', 'Linux'])
('Updated List:', ['Linux', 'macOS', 'Windows'])

    \end{Verbatim}

    Printing Elements in Reversed Order

reveresed() is a function that allows accessing a list in reverse order

    \begin{Verbatim}[commandchars=\\\{\}]
{\color{incolor}In [{\color{incolor}72}]:} \PY{k}{for} \PY{n}{o} \PY{o+ow}{in} \PY{n+nb}{reversed}\PY{p}{(}\PY{n}{os}\PY{p}{)}\PY{p}{:}
             \PY{k}{print}\PY{p}{(}\PY{n}{o}\PY{p}{)}
\end{Verbatim}


    \begin{Verbatim}[commandchars=\\\{\}]
Linux
macOS
Windows

    \end{Verbatim}

    \subparagraph{sort(), copy() and clear()}\label{sort-copy-and-clear}

Sort() can be used to sort the elements of a list.

Syntax: list.sort(key=..., reverse=...)

    \begin{Verbatim}[commandchars=\\\{\}]
{\color{incolor}In [{\color{incolor}76}]:} \PY{n}{vowels} \PY{o}{=} \PY{p}{[}\PY{l+s+s1}{\PYZsq{}}\PY{l+s+s1}{e}\PY{l+s+s1}{\PYZsq{}}\PY{p}{,} \PY{l+s+s1}{\PYZsq{}}\PY{l+s+s1}{a}\PY{l+s+s1}{\PYZsq{}}\PY{p}{,} \PY{l+s+s1}{\PYZsq{}}\PY{l+s+s1}{u}\PY{l+s+s1}{\PYZsq{}}\PY{p}{,} \PY{l+s+s1}{\PYZsq{}}\PY{l+s+s1}{o}\PY{l+s+s1}{\PYZsq{}}\PY{p}{,} \PY{l+s+s1}{\PYZsq{}}\PY{l+s+s1}{i}\PY{l+s+s1}{\PYZsq{}}\PY{p}{]}
         \PY{n}{vowels}\PY{o}{.}\PY{n}{sort}\PY{p}{(}\PY{p}{)}
         
         \PY{k}{print}\PY{p}{(}\PY{l+s+s1}{\PYZsq{}}\PY{l+s+s1}{Sorted list:}\PY{l+s+s1}{\PYZsq{}}\PY{p}{,} \PY{n}{vowels}\PY{p}{)} \PY{c+c1}{\PYZsh{}sorts the ASCII values}
         
         \PY{n}{vowels}\PY{o}{.}\PY{n}{sort}\PY{p}{(}\PY{n}{reverse}\PY{o}{=}\PY{n+nb+bp}{True}\PY{p}{)} \PY{c+c1}{\PYZsh{}sort in reverse order}
         \PY{k}{print}\PY{p}{(}\PY{l+s+s1}{\PYZsq{}}\PY{l+s+s1}{Sorted list (in Descending):}\PY{l+s+s1}{\PYZsq{}}\PY{p}{,} \PY{n}{vowels}\PY{p}{)}
\end{Verbatim}


    \begin{Verbatim}[commandchars=\\\{\}]
('Sorted list:', ['a', 'e', 'i', 'o', 'u'])
('Sorted list (in Descending):', ['u', 'o', 'i', 'e', 'a'])

    \end{Verbatim}

    Copy() can be used to copy one list to another and clear() can be used
to remove all items from a list.

    \begin{Verbatim}[commandchars=\\\{\}]
{\color{incolor}In [{\color{incolor}82}]:} \PY{n}{old\PYZus{}list} \PY{o}{=} \PY{p}{[}\PY{l+s+s1}{\PYZsq{}}\PY{l+s+s1}{cat}\PY{l+s+s1}{\PYZsq{}}\PY{p}{,} \PY{l+m+mi}{0}\PY{p}{,} \PY{l+m+mf}{6.7}\PY{p}{]}
         \PY{n}{new\PYZus{}list} \PY{o}{=} \PY{n}{old\PYZus{}list}\PY{o}{.}\PY{n}{copy}\PY{p}{(}\PY{p}{)} \PY{c+c1}{\PYZsh{}NEED TO UNDERSTAND THIS ERROR (NOTE FOR SELF)}
         \PY{n}{new\PYZus{}list}\PY{o}{.}\PY{n}{append}\PY{p}{(}\PY{l+s+s1}{\PYZsq{}}\PY{l+s+s1}{dog}\PY{l+s+s1}{\PYZsq{}}\PY{p}{)}
         
         \PY{c+c1}{\PYZsh{}same thing can be achieved using the = operator}
         \PY{n}{newer\PYZus{}list} \PY{o}{=} \PY{n}{old\PYZus{}list}
         
         \PY{c+c1}{\PYZsh{} Printing new and old list}
         \PY{k}{print}\PY{p}{(}\PY{n}{new\PYZus{}list}\PY{p}{)}
         \PY{k}{print}\PY{p}{(}\PY{n}{newer\PYZus{}list}\PY{p}{)}
\end{Verbatim}


    \begin{Verbatim}[commandchars=\\\{\}]

        

        AttributeErrorTraceback (most recent call last)

        <ipython-input-82-88bfa74e3e4f> in <module>()
          1 old\_list = ['cat', 0, 6.7]
    ----> 2 new\_list = old\_list.copy()
          3 new\_list.append('dog')
          4 
          5 \#same thing can be achieved using the = operator
    

        AttributeError: 'list' object has no attribute 'copy'

    \end{Verbatim}

    Clear() can be used to remove all elements from the list

    \begin{Verbatim}[commandchars=\\\{\}]
{\color{incolor}In [{\color{incolor}83}]:} \PY{k}{print}\PY{p}{(}\PY{n+nb}{len}\PY{p}{(}\PY{n}{old\PYZus{}list}\PY{o}{.}\PY{n}{clear}\PY{p}{(}\PY{p}{)}\PY{p}{)}\PY{p}{)}
         \PY{k}{print}\PY{p}{(}\PY{n}{old\PYZus{}list}\PY{p}{)}
\end{Verbatim}


    \begin{Verbatim}[commandchars=\\\{\}]

        

        AttributeErrorTraceback (most recent call last)

        <ipython-input-83-59af470159c4> in <module>()
    ----> 1 print(len(old\_list.clear()))
          2 print(old\_list)
    

        AttributeError: 'list' object has no attribute 'clear'

    \end{Verbatim}

    \subsection{For Loop with Lists and
Tuples}\label{for-loop-with-lists-and-tuples}

    A for loop is used to run through the elements of a list or a tuple.

    \begin{Verbatim}[commandchars=\\\{\}]
{\color{incolor}In [{\color{incolor}19}]:} \PY{c+c1}{\PYZsh{}Iterationg through a list using for Loop}
         \PY{n}{shopping} \PY{o}{=} \PY{p}{[}\PY{l+s+s1}{\PYZsq{}}\PY{l+s+s1}{bread}\PY{l+s+s1}{\PYZsq{}}\PY{p}{,}\PY{l+s+s1}{\PYZsq{}}\PY{l+s+s1}{jam}\PY{l+s+s1}{\PYZsq{}}\PY{p}{,}\PY{l+s+s1}{\PYZsq{}}\PY{l+s+s1}{butter}\PY{l+s+s1}{\PYZsq{}}\PY{p}{]}
         \PY{k}{for} \PY{n}{item} \PY{o+ow}{in} \PY{n}{shopping}\PY{p}{:}
             \PY{k}{print}\PY{p}{(}\PY{n}{item} \PY{o}{+} \PY{l+s+s1}{\PYZsq{}}\PY{l+s+s1}{ has been added to the cart}\PY{l+s+s1}{\PYZsq{}}\PY{p}{)}
             
         \PY{k}{print}
         
         \PY{c+c1}{\PYZsh{}Iterating through a tuple using for loop}
         \PY{n}{days\PYZus{}in\PYZus{}a\PYZus{}week} \PY{o}{=} \PY{p}{(}\PY{l+s+s1}{\PYZsq{}}\PY{l+s+s1}{Sunday}\PY{l+s+s1}{\PYZsq{}}\PY{p}{,}\PY{l+s+s1}{\PYZsq{}}\PY{l+s+s1}{Monday}\PY{l+s+s1}{\PYZsq{}}\PY{p}{,}\PY{l+s+s1}{\PYZsq{}}\PY{l+s+s1}{Tuesday}\PY{l+s+s1}{\PYZsq{}}\PY{p}{,}\PY{l+s+s1}{\PYZsq{}}\PY{l+s+s1}{Wednesday}\PY{l+s+s1}{\PYZsq{}}\PY{p}{,}\PY{l+s+s1}{\PYZsq{}}\PY{l+s+s1}{Thursday}\PY{l+s+s1}{\PYZsq{}}\PY{p}{,}\PY{l+s+s1}{\PYZsq{}}\PY{l+s+s1}{Friday}\PY{l+s+s1}{\PYZsq{}}\PY{p}{,}\PY{l+s+s1}{\PYZsq{}}\PY{l+s+s1}{Saturday}\PY{l+s+s1}{\PYZsq{}}\PY{p}{)}
         \PY{k}{for} \PY{n}{a\PYZus{}day} \PY{o+ow}{in} \PY{n}{days\PYZus{}in\PYZus{}a\PYZus{}week}\PY{p}{:}
             \PY{k}{print} \PY{p}{(}\PY{n}{a\PYZus{}day}\PY{p}{)}
\end{Verbatim}


    \begin{Verbatim}[commandchars=\\\{\}]
bread has been added to the cart
jam has been added to the cart
butter has been added to the cart

Sunday
Monday
Tuesday
Wednesday
Thursday
Friday
Saturday

    \end{Verbatim}

    \begin{Verbatim}[commandchars=\\\{\}]
{\color{incolor}In [{\color{incolor}21}]:} \PY{c+c1}{\PYZsh{}Iterating through a List without a name that is defined inside the loop only}
         \PY{k}{for} \PY{n}{num} \PY{o+ow}{in} \PY{p}{[}\PY{l+m+mi}{1}\PY{p}{,}\PY{l+m+mi}{2}\PY{p}{,}\PY{l+m+mi}{3}\PY{p}{,}\PY{l+m+mi}{4}\PY{p}{,}\PY{l+m+mi}{5}\PY{p}{]}\PY{p}{:} 
             \PY{k}{print} \PY{p}{(}\PY{l+s+s1}{\PYZsq{}}\PY{l+s+s1}{The square of }\PY{l+s+s1}{\PYZsq{}}\PY{o}{+} \PY{n+nb}{str}\PY{p}{(}\PY{n}{num}\PY{p}{)} \PY{o}{+} \PY{l+s+s1}{\PYZsq{}}\PY{l+s+s1}{ is }\PY{l+s+s1}{\PYZsq{}}\PY{o}{+} \PY{n+nb}{str}\PY{p}{(}\PY{n}{num}\PY{o}{*}\PY{o}{*}\PY{l+m+mi}{2}\PY{p}{)}\PY{p}{)}
         
         \PY{k}{print}
         
         \PY{c+c1}{\PYZsh{}Iterating through a tuple without a name that is defined inside the loop only}
         \PY{k}{for} \PY{n}{even\PYZus{}num} \PY{o+ow}{in} \PY{p}{(}\PY{l+m+mi}{2}\PY{p}{,}\PY{l+m+mi}{4}\PY{p}{,}\PY{l+m+mi}{6}\PY{p}{,}\PY{l+m+mi}{8}\PY{p}{,}\PY{l+m+mi}{10}\PY{p}{)}\PY{p}{:} 
             \PY{k}{print} \PY{p}{(}\PY{l+s+s1}{\PYZsq{}}\PY{l+s+s1}{The half of }\PY{l+s+s1}{\PYZsq{}}\PY{o}{+} \PY{n+nb}{str}\PY{p}{(}\PY{n}{even\PYZus{}num}\PY{p}{)} \PY{o}{+} \PY{l+s+s1}{\PYZsq{}}\PY{l+s+s1}{ is }\PY{l+s+s1}{\PYZsq{}}\PY{o}{+} \PY{n+nb}{str}\PY{p}{(}\PY{n}{even\PYZus{}num}\PY{o}{/}\PY{l+m+mi}{2}\PY{p}{)}\PY{p}{)}
\end{Verbatim}


    \begin{Verbatim}[commandchars=\\\{\}]
The square of 1 is 1
The square of 2 is 4
The square of 3 is 9
The square of 4 is 16
The square of 5 is 25

The half of 2 is 1
The half of 4 is 2
The half of 6 is 3
The half of 8 is 4
The half of 10 is 5

    \end{Verbatim}

    We can also loop using the indices of the Lists

    \begin{Verbatim}[commandchars=\\\{\}]
{\color{incolor}In [{\color{incolor}23}]:} \PY{n}{names} \PY{o}{=} \PY{p}{[}\PY{l+s+s1}{\PYZsq{}}\PY{l+s+s1}{Adam}\PY{l+s+s1}{\PYZsq{}}\PY{p}{,}\PY{l+s+s1}{\PYZsq{}}\PY{l+s+s1}{Eve}\PY{l+s+s1}{\PYZsq{}}\PY{p}{,}\PY{l+s+s1}{\PYZsq{}}\PY{l+s+s1}{Jack}\PY{l+s+s1}{\PYZsq{}}\PY{p}{,}\PY{l+s+s1}{\PYZsq{}}\PY{l+s+s1}{Jill}\PY{l+s+s1}{\PYZsq{}}\PY{p}{]}
         \PY{n}{i}\PY{o}{=}\PY{l+m+mi}{0}
         \PY{k}{for} \PY{n}{i}  \PY{o+ow}{in} \PY{n+nb}{range} \PY{p}{(}\PY{l+m+mi}{0}\PY{p}{,} \PY{l+m+mi}{4}\PY{p}{)}\PY{p}{:}
             \PY{k}{print} \PY{p}{(}\PY{n}{names}\PY{p}{[}\PY{n}{i}\PY{p}{]}\PY{p}{)}  
             \PY{n}{i} \PY{o}{=} \PY{n}{i} \PY{o}{+} \PY{l+m+mi}{1}
             
         \PY{n}{colors} \PY{o}{=} \PY{p}{(}\PY{l+s+s1}{\PYZsq{}}\PY{l+s+s1}{red}\PY{l+s+s1}{\PYZsq{}}\PY{p}{,}\PY{l+s+s1}{\PYZsq{}}\PY{l+s+s1}{black}\PY{l+s+s1}{\PYZsq{}}\PY{p}{,}\PY{l+s+s1}{\PYZsq{}}\PY{l+s+s1}{green}\PY{l+s+s1}{\PYZsq{}}\PY{p}{,}\PY{l+s+s1}{\PYZsq{}}\PY{l+s+s1}{blue}\PY{l+s+s1}{\PYZsq{}}\PY{p}{)}
         \PY{n}{k} \PY{o}{=} \PY{l+m+mi}{0}
         \PY{k}{for} \PY{n}{k} \PY{o+ow}{in} \PY{n+nb}{range} \PY{p}{(}\PY{l+m+mi}{0}\PY{p}{,}\PY{l+m+mi}{4}\PY{p}{)}\PY{p}{:}
             \PY{k}{print} \PY{p}{(}\PY{n}{colors}\PY{p}{[}\PY{n}{k}\PY{p}{]}\PY{p}{)}
\end{Verbatim}


    \begin{Verbatim}[commandchars=\\\{\}]
Adam
Eve
Jack
Jill
red
black
green
blue

    \end{Verbatim}

    \subsection{If Statement with Lists and
Tuples}\label{if-statement-with-lists-and-tuples}

    We can compare the elements of lists and tuples using their indices with
IF Statement.

    \begin{Verbatim}[commandchars=\\\{\}]
{\color{incolor}In [{\color{incolor}28}]:} \PY{n}{shopping\PYZus{}list} \PY{o}{=} \PY{p}{[}\PY{l+s+s1}{\PYZsq{}}\PY{l+s+s1}{bread}\PY{l+s+s1}{\PYZsq{}}\PY{p}{,}\PY{l+s+s1}{\PYZsq{}}\PY{l+s+s1}{jam}\PY{l+s+s1}{\PYZsq{}}\PY{p}{,}\PY{l+s+s1}{\PYZsq{}}\PY{l+s+s1}{marmite}\PY{l+s+s1}{\PYZsq{}}\PY{p}{]}
         
         \PY{k}{for} \PY{n}{item} \PY{o+ow}{in} \PY{n}{shopping\PYZus{}list}\PY{p}{:}
             \PY{k}{if} \PY{n}{item} \PY{o}{==} \PY{l+s+s1}{\PYZsq{}}\PY{l+s+s1}{marmite}\PY{l+s+s1}{\PYZsq{}}\PY{p}{:}
                 \PY{k}{print}\PY{p}{(}\PY{l+s+s1}{\PYZsq{}}\PY{l+s+s1}{Yuck!!}\PY{l+s+s1}{\PYZsq{}}\PY{p}{)}
             \PY{k}{else}\PY{p}{:}
                 \PY{k}{print}\PY{p}{(}\PY{l+s+s1}{\PYZsq{}}\PY{l+s+s1}{Yum!!!}\PY{l+s+s1}{\PYZsq{}}\PY{p}{)}
\end{Verbatim}


    \begin{Verbatim}[commandchars=\\\{\}]
Yum!!!
Yum!!!
Yuck!!

    \end{Verbatim}

    \begin{Verbatim}[commandchars=\\\{\}]
{\color{incolor}In [{\color{incolor}30}]:} \PY{k}{if} \PY{n}{shopping\PYZus{}list}\PY{p}{[}\PY{l+m+mi}{0}\PY{p}{]} \PY{o}{==} \PY{l+s+s1}{\PYZsq{}}\PY{l+s+s1}{bread}\PY{l+s+s1}{\PYZsq{}}\PY{p}{:} \PY{c+c1}{\PYZsh{}using index to compare the element inside IF statement}
             \PY{k}{print}\PY{p}{(}\PY{l+s+s1}{\PYZsq{}}\PY{l+s+s1}{Bread is awesome}\PY{l+s+s1}{\PYZsq{}}\PY{p}{)} 
         \PY{k}{else}\PY{p}{:}
             \PY{k}{print}\PY{p}{(}\PY{l+s+s1}{\PYZsq{}}\PY{l+s+s1}{Jam and Marmite are awesome as well! Oh wait, Marmite is not.}\PY{l+s+s1}{\PYZsq{}}\PY{p}{)}
\end{Verbatim}


    \begin{Verbatim}[commandchars=\\\{\}]
Bread is awesome

    \end{Verbatim}

    The usage of If Statement and Tuples is similar.

    \begin{Verbatim}[commandchars=\\\{\}]
{\color{incolor}In [{\color{incolor}31}]:} \PY{n}{weekdays} \PY{o}{=} \PY{p}{(}\PY{l+s+s1}{\PYZsq{}}\PY{l+s+s1}{Monday}\PY{l+s+s1}{\PYZsq{}}\PY{p}{,}\PY{l+s+s1}{\PYZsq{}}\PY{l+s+s1}{Tuesday}\PY{l+s+s1}{\PYZsq{}}\PY{p}{,}\PY{l+s+s1}{\PYZsq{}}\PY{l+s+s1}{Wednesday}\PY{l+s+s1}{\PYZsq{}}\PY{p}{,}\PY{l+s+s1}{\PYZsq{}}\PY{l+s+s1}{Thursday}\PY{l+s+s1}{\PYZsq{}}\PY{p}{,}\PY{l+s+s1}{\PYZsq{}}\PY{l+s+s1}{Friday}\PY{l+s+s1}{\PYZsq{}}\PY{p}{)}
         \PY{k}{for} \PY{n}{day} \PY{o+ow}{in} \PY{n}{weekdays}\PY{p}{:}
             \PY{k}{if} \PY{n}{day} \PY{o}{==} \PY{l+s+s1}{\PYZsq{}}\PY{l+s+s1}{Friday}\PY{l+s+s1}{\PYZsq{}}\PY{p}{:}
                 \PY{k}{print}\PY{p}{(}\PY{l+s+s1}{\PYZsq{}}\PY{l+s+s1}{Thank God!}\PY{l+s+s1}{\PYZsq{}}\PY{p}{)}
             \PY{k}{else}\PY{p}{:}
                 \PY{k}{print}\PY{p}{(}\PY{l+s+s1}{\PYZsq{}}\PY{l+s+s1}{Waiting for Friday!}\PY{l+s+s1}{\PYZsq{}}\PY{p}{)}
\end{Verbatim}


    \begin{Verbatim}[commandchars=\\\{\}]
Waiting for Friday!
Waiting for Friday!
Waiting for Friday!
Waiting for Friday!
Thank God!

    \end{Verbatim}

    \subsubsection{While Loops with Lists and
Tuples}\label{while-loops-with-lists-and-tuples}

If we wanted to mimic the behavior of our traditional C-style for loop
in Python, we could use a while loop.

    \begin{Verbatim}[commandchars=\\\{\}]
{\color{incolor}In [{\color{incolor}32}]:} \PY{n}{colors} \PY{o}{=} \PY{p}{[}\PY{l+s+s2}{\PYZdq{}}\PY{l+s+s2}{red}\PY{l+s+s2}{\PYZdq{}}\PY{p}{,} \PY{l+s+s2}{\PYZdq{}}\PY{l+s+s2}{green}\PY{l+s+s2}{\PYZdq{}}\PY{p}{,} \PY{l+s+s2}{\PYZdq{}}\PY{l+s+s2}{blue}\PY{l+s+s2}{\PYZdq{}}\PY{p}{,} \PY{l+s+s2}{\PYZdq{}}\PY{l+s+s2}{purple}\PY{l+s+s2}{\PYZdq{}}\PY{p}{]}
         \PY{n}{i} \PY{o}{=} \PY{l+m+mi}{0}
         \PY{k}{while} \PY{n}{i} \PY{o}{\PYZlt{}} \PY{n+nb}{len}\PY{p}{(}\PY{n}{colors}\PY{p}{)}\PY{p}{:}
             \PY{k}{print}\PY{p}{(}\PY{n}{colors}\PY{p}{[}\PY{n}{i}\PY{p}{]}\PY{p}{)}
             \PY{n}{i} \PY{o}{+}\PY{o}{=} \PY{l+m+mi}{1}
\end{Verbatim}


    \begin{Verbatim}[commandchars=\\\{\}]
red
green
blue
purple

    \end{Verbatim}

    \begin{Verbatim}[commandchars=\\\{\}]
{\color{incolor}In [{\color{incolor}33}]:} \PY{n}{colors} \PY{o}{=} \PY{p}{[}\PY{l+s+s2}{\PYZdq{}}\PY{l+s+s2}{red}\PY{l+s+s2}{\PYZdq{}}\PY{p}{,} \PY{l+s+s2}{\PYZdq{}}\PY{l+s+s2}{green}\PY{l+s+s2}{\PYZdq{}}\PY{p}{,} \PY{l+s+s2}{\PYZdq{}}\PY{l+s+s2}{blue}\PY{l+s+s2}{\PYZdq{}}\PY{p}{,} \PY{l+s+s2}{\PYZdq{}}\PY{l+s+s2}{purple}\PY{l+s+s2}{\PYZdq{}}\PY{p}{]}
         \PY{k}{while} \PY{n}{i} \PY{o+ow}{in} \PY{n+nb}{range}\PY{p}{(}\PY{n+nb}{len}\PY{p}{(}\PY{n}{colors}\PY{p}{)}\PY{p}{)}\PY{p}{:}
             \PY{k}{print}\PY{p}{(}\PY{n}{colors}\PY{p}{[}\PY{n}{i}\PY{p}{]}\PY{p}{)}
\end{Verbatim}


    \begin{Verbatim}[commandchars=\\\{\}]
red
green
blue
purple

    \end{Verbatim}

    \begin{Verbatim}[commandchars=\\\{\}]
{\color{incolor}In [{\color{incolor}37}]:} \PY{c+c1}{\PYZsh{}when we have to use indices.. same format can be applied to for loop as well}
         \PY{n}{presidents} \PY{o}{=} \PY{p}{[}\PY{l+s+s2}{\PYZdq{}}\PY{l+s+s2}{Washington}\PY{l+s+s2}{\PYZdq{}}\PY{p}{,} \PY{l+s+s2}{\PYZdq{}}\PY{l+s+s2}{Adams}\PY{l+s+s2}{\PYZdq{}}\PY{p}{,} \PY{l+s+s2}{\PYZdq{}}\PY{l+s+s2}{Jefferson}\PY{l+s+s2}{\PYZdq{}}\PY{p}{,} \PY{l+s+s2}{\PYZdq{}}\PY{l+s+s2}{Madison}\PY{l+s+s2}{\PYZdq{}}\PY{p}{,} \PY{l+s+s2}{\PYZdq{}}\PY{l+s+s2}{Monroe}\PY{l+s+s2}{\PYZdq{}}\PY{p}{,} \PY{l+s+s2}{\PYZdq{}}\PY{l+s+s2}{Adams}\PY{l+s+s2}{\PYZdq{}}\PY{p}{,} \PY{l+s+s2}{\PYZdq{}}\PY{l+s+s2}{Jackson}\PY{l+s+s2}{\PYZdq{}}\PY{p}{]}
         \PY{n}{i} \PY{o}{=}\PY{l+m+mi}{0}
         \PY{k}{while} \PY{n}{i} \PY{o+ow}{in} \PY{n+nb}{range}\PY{p}{(}\PY{n+nb}{len}\PY{p}{(}\PY{n}{presidents}\PY{p}{)}\PY{p}{)}\PY{p}{:}
             \PY{k}{print}\PY{p}{(}\PY{l+s+s2}{\PYZdq{}}\PY{l+s+s2}{President \PYZob{}\PYZcb{}: \PYZob{}\PYZcb{}}\PY{l+s+s2}{\PYZdq{}}\PY{o}{.}\PY{n}{format}\PY{p}{(}\PY{n}{i} \PY{o}{+} \PY{l+m+mi}{1}\PY{p}{,} \PY{n}{presidents}\PY{p}{[}\PY{n}{i}\PY{p}{]}\PY{p}{)}\PY{p}{)}
             \PY{n}{i}\PY{o}{+}\PY{o}{=}\PY{l+m+mi}{1}
\end{Verbatim}


    \begin{Verbatim}[commandchars=\\\{\}]
President 1: Washington
President 2: Adams
President 3: Jefferson
President 4: Madison
President 5: Monroe
President 6: Adams
President 7: Jackson

    \end{Verbatim}

    \subsection{MORE on Looping with Lists and
Tuples}\label{more-on-looping-with-lists-and-tuples}

Python's built-in enumerate function allows us to loop over a list and
retrieve both the index and the value of each item in the list. The
enumerate function gives us an iterable where each element is a tuple
that contains the index of the item and the original item value.

This function is meant for solving the task of: 1. Accessing each item
in a list (or another iterable) 2. Also getting the index of each item
accessed

    \begin{Verbatim}[commandchars=\\\{\}]
{\color{incolor}In [{\color{incolor}38}]:} \PY{n}{colors} \PY{o}{=} \PY{p}{[}\PY{l+s+s2}{\PYZdq{}}\PY{l+s+s2}{red}\PY{l+s+s2}{\PYZdq{}}\PY{p}{,} \PY{l+s+s2}{\PYZdq{}}\PY{l+s+s2}{green}\PY{l+s+s2}{\PYZdq{}}\PY{p}{,} \PY{l+s+s2}{\PYZdq{}}\PY{l+s+s2}{blue}\PY{l+s+s2}{\PYZdq{}}\PY{p}{,} \PY{l+s+s2}{\PYZdq{}}\PY{l+s+s2}{purple}\PY{l+s+s2}{\PYZdq{}}\PY{p}{]}
         \PY{n}{ratios} \PY{o}{=} \PY{p}{[}\PY{l+m+mf}{0.2}\PY{p}{,} \PY{l+m+mf}{0.3}\PY{p}{,} \PY{l+m+mf}{0.1}\PY{p}{,} \PY{l+m+mf}{0.4}\PY{p}{]}
         \PY{k}{for} \PY{n}{i}\PY{p}{,} \PY{n}{color} \PY{o+ow}{in} \PY{n+nb}{enumerate}\PY{p}{(}\PY{n}{colors}\PY{p}{)}\PY{p}{:} \PY{c+c1}{\PYZsh{}works in the same way with tuples}
             \PY{n}{ratio} \PY{o}{=} \PY{n}{ratios}\PY{p}{[}\PY{n}{i}\PY{p}{]}
             \PY{k}{print}\PY{p}{(}\PY{l+s+s2}{\PYZdq{}}\PY{l+s+s2}{\PYZob{}\PYZcb{}}\PY{l+s+s2}{\PYZpc{}}\PY{l+s+s2}{ \PYZob{}\PYZcb{}}\PY{l+s+s2}{\PYZdq{}}\PY{o}{.}\PY{n}{format}\PY{p}{(}\PY{n}{ratio} \PY{o}{*} \PY{l+m+mi}{100}\PY{p}{,} \PY{n}{color}\PY{p}{)}\PY{p}{)}
\end{Verbatim}


    \begin{Verbatim}[commandchars=\\\{\}]
20.0\% red
30.0\% green
10.0\% blue
40.0\% purple

    \end{Verbatim}

    We can use zip function to iterate over mulitple lists at the same time

    \begin{Verbatim}[commandchars=\\\{\}]
{\color{incolor}In [{\color{incolor}42}]:} \PY{n}{colors} \PY{o}{=} \PY{p}{(}\PY{l+s+s2}{\PYZdq{}}\PY{l+s+s2}{red}\PY{l+s+s2}{\PYZdq{}}\PY{p}{,} \PY{l+s+s2}{\PYZdq{}}\PY{l+s+s2}{green}\PY{l+s+s2}{\PYZdq{}}\PY{p}{,} \PY{l+s+s2}{\PYZdq{}}\PY{l+s+s2}{blue}\PY{l+s+s2}{\PYZdq{}}\PY{p}{,} \PY{l+s+s2}{\PYZdq{}}\PY{l+s+s2}{purple}\PY{l+s+s2}{\PYZdq{}}\PY{p}{)} 
         \PY{n}{ratios} \PY{o}{=} \PY{p}{(}\PY{l+m+mf}{0.2}\PY{p}{,} \PY{l+m+mf}{0.3}\PY{p}{,} \PY{l+m+mf}{0.1}\PY{p}{,} \PY{l+m+mf}{0.4}\PY{p}{)}
         \PY{k}{for} \PY{n}{color}\PY{p}{,} \PY{n}{ratio} \PY{o+ow}{in} \PY{n+nb}{zip}\PY{p}{(}\PY{n}{colors}\PY{p}{,} \PY{n}{ratios}\PY{p}{)}\PY{p}{:}\PY{c+c1}{\PYZsh{}works in the same way for lists}
             \PY{k}{print}\PY{p}{(}\PY{l+s+s2}{\PYZdq{}}\PY{l+s+s2}{\PYZob{}\PYZcb{}}\PY{l+s+s2}{\PYZpc{}}\PY{l+s+s2}{ \PYZob{}\PYZcb{}}\PY{l+s+s2}{\PYZdq{}}\PY{o}{.}\PY{n}{format}\PY{p}{(}\PY{n}{ratio} \PY{o}{*} \PY{l+m+mi}{100}\PY{p}{,} \PY{n}{color}\PY{p}{)}\PY{p}{)}
\end{Verbatim}


    \begin{Verbatim}[commandchars=\\\{\}]
20.0\% red
30.0\% green
10.0\% blue
40.0\% purple

    \end{Verbatim}

    \subsection{Lists and Tuples with
Lambda}\label{lists-and-tuples-with-lambda}

    \begin{Verbatim}[commandchars=\\\{\}]
{\color{incolor}In [{\color{incolor}46}]:} \PY{n}{x} \PY{o}{=} \PY{p}{[}\PY{l+m+mi}{2}\PY{p}{,} \PY{l+m+mi}{3}\PY{p}{,} \PY{l+m+mi}{4}\PY{p}{,} \PY{l+m+mi}{5}\PY{p}{,} \PY{l+m+mi}{6}\PY{p}{]}
         \PY{n}{y} \PY{o}{=} \PY{p}{[}\PY{p}{]}
         \PY{k}{for} \PY{n}{v} \PY{o+ow}{in} \PY{n}{x}\PY{p}{:}
             \PY{n}{y} \PY{o}{+}\PY{o}{=} \PY{p}{[}\PY{n}{v} \PY{o}{*} \PY{l+m+mi}{5}\PY{p}{]}
         \PY{k}{print} \PY{n}{x}
         \PY{k}{print} \PY{n}{y}
\end{Verbatim}


    \begin{Verbatim}[commandchars=\\\{\}]
[2, 3, 4, 5, 6]
[10, 15, 20, 25, 30]

    \end{Verbatim}

    Above result can be obtained using lamda operator and the map function

    \begin{Verbatim}[commandchars=\\\{\}]
{\color{incolor}In [{\color{incolor}48}]:} \PY{n}{y} \PY{o}{=} \PY{n+nb}{map}\PY{p}{(}\PY{k}{lambda} \PY{n}{v} \PY{p}{:} \PY{n}{v} \PY{o}{*} \PY{l+m+mi}{5}\PY{p}{,} \PY{n}{x}\PY{p}{)}
         \PY{k}{print} \PY{n}{y}
\end{Verbatim}


    \begin{Verbatim}[commandchars=\\\{\}]
[10, 15, 20, 25, 30]

    \end{Verbatim}

    \begin{Verbatim}[commandchars=\\\{\}]
{\color{incolor}In [{\color{incolor}52}]:} \PY{n}{x} \PY{o}{=} \PY{p}{(}\PY{l+m+mi}{1}\PY{p}{,}\PY{l+m+mi}{2}\PY{p}{,}\PY{l+m+mi}{3}\PY{p}{,}\PY{l+m+mi}{4}\PY{p}{,}\PY{l+m+mi}{5}\PY{p}{)} \PY{c+c1}{\PYZsh{}TUPLE}
         \PY{n}{y} \PY{o}{=} \PY{n+nb}{map}\PY{p}{(}\PY{k}{lambda} \PY{n}{v} \PY{p}{:} \PY{n}{v} \PY{o}{*} \PY{l+m+mi}{5}\PY{p}{,} \PY{n}{x}\PY{p}{)} 
         \PY{k}{print} \PY{n}{y} \PY{c+c1}{\PYZsh{}same action on a tuple yields a List as the map function returns a List}
\end{Verbatim}


    \begin{Verbatim}[commandchars=\\\{\}]
[5, 10, 15, 20, 25]

    \end{Verbatim}

    More complex operations can be achieved using Lambdas

    \begin{Verbatim}[commandchars=\\\{\}]
{\color{incolor}In [{\color{incolor}56}]:} \PY{n}{x} \PY{o}{=} \PY{p}{[}\PY{l+m+mi}{2}\PY{p}{,} \PY{l+m+mi}{3}\PY{p}{,} \PY{l+m+mi}{4}\PY{p}{,} \PY{l+m+mi}{5}\PY{p}{,} \PY{l+m+mi}{6}\PY{p}{]}
         \PY{n}{y} \PY{o}{=} \PY{n+nb}{map}\PY{p}{(}\PY{k}{lambda} \PY{n}{v} \PY{p}{:} \PY{n}{v} \PY{o}{*} \PY{l+m+mi}{5}\PY{p}{,} \PY{n+nb}{filter}\PY{p}{(}\PY{k}{lambda} \PY{n}{u} \PY{p}{:} \PY{n}{u} \PY{o}{\PYZpc{}} \PY{l+m+mi}{2}\PY{p}{,} \PY{n}{x}\PY{p}{)}\PY{p}{)}
         
         \PY{k}{print} \PY{p}{(}\PY{n+nb}{filter}\PY{p}{(}\PY{k}{lambda} \PY{n}{u}\PY{p}{:} \PY{n}{u} \PY{o}{\PYZpc{}} \PY{l+m+mi}{2}\PY{p}{,} \PY{n}{x}\PY{p}{)}\PY{p}{)}
         \PY{c+c1}{\PYZsh{}select the odd numbers number from the list x, and then multiply it by 5}
         \PY{k}{print} \PY{n}{y}
\end{Verbatim}


    \begin{Verbatim}[commandchars=\\\{\}]
[3, 5]
[15, 25]

    \end{Verbatim}


    % Add a bibliography block to the postdoc
    
    
    
    \end{document}
